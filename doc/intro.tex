\chapter{Různé}

\section[0]{Notation}
Let $\Form_\Var$ be the set of all prepositional formulas over
  the set of variables $\Var$;
  $\Val_\Var$ be the set of all valuations of variables $\Var$.
Formulas $\form_0, \form_1 \in \Form_\Var$ are (semantically) equivalent,
  written $\form_0 \equiv \form_1$, if
  $\val(\form_0) = \val(\form_1)$ for all $\val\in\Val_\Var$.
For a formula $\form\in\Form_\Var$, let
  $\numval_\Var(\form) = |\{ \val\in\Val_\Var \| \val(\form) = 1 \}|$
  be the number of valuations by which $\form$ is satisfied.
We often omit the index $\Var$ if it is clear from the context.

For any unary predicate $P$, $\#i\in A.P(i) = |\{ i\in A \| P(i)\}|$.
  We often omit the ``$\in A$'' part and write only $\#i.P(i)$
  if the range of $i$ is clear from the context.

Let $\Perm_\Var$ be the set of all permutations of $\Var$.

\section{Fake-coin problem}

There is a lot of variants of a logic puzzle with coins and a pair of scales balance.
Here we present the most interesting ones and their generalization, which we study in the sequel.

In all the problems, you can use the scales only to weight coins.
You can put as many coins at the sides as you like as long as the number is the same.
All information you get is that both sides weight equally or which side is heavier (i.e., there are 3 possible results).

The weight of a \emph{fake} coin is always different than
the weight of a authentic one
but it is not know whether it is heavier or ligther.

\begin{problem}[The twelve coin problem]
You are given 12 coins, exactly one of which is fake.
Determine the unique coin and its weight relavite to others.
You can use the balance at most 3 times.
\end{problem}

\begin{problem}[The thirteen coin problem]
You are given 13 coins, exactly one of which is fake.
You have one more coin at your disposal which is guaranteed to be authentic.
Determine the unique fake coin and its weight relavite to others.
\end{problem}

\begin{problem}[General fake coin problem]
You are given $n$ coins, $f$ of them are fake (some of them may be lighter, some of them heavier).
You have another $m$ authentic coins at your disposal.
In as less weightings as possible, determine which coins are fake.
\end{problem}

\section{Mastermind}

\emph{Mastermind} is a classic 2-player board game invented by Mordecai Meirowitz in 1970[wiki].
The principle of the game is the same as of \emph{Bulls and Cows}, it just uses colors instead of letters.




\chapter{Implementation details}

\section{Programming Language and Style}

Since the problem we are trying to solve is very computationally demanding,
  we had to choose a high-performing programming language.
The tools we want to use, especially SAT solvers, are typically written in C/C++,
  so C++ was a natural choice for our tool.
Cobra is written in the latest standard of ISO C++, namely C++11, which
  contains significant changes both in the language and in the standard libraries
  and, in our opinion, improves readability compared to previous versions.

We wanted the style of our code to be consistent and to usage of the language in the best
 manner possible according to industrial practice.
From the wide range of style guides available online
 we chose \emph{Google C++ Style Guide}\ref{google-style} and made
 the code compliant with all its rules except for a few exception.
The only significant one of those are lambda functions, which are forbidden
 by the style guide due to various reasons\ref{google-style},
 but we think they are more beneficial than harmful in this project.

\subsection{Compiler Requirements}
The usage of a modern standard requires a modern compiler,
  which supports all the C++11 feautres we use.
We recommend using starndard \texttt{gcc}; you need version $4.8$ or higher.
For \texttt{clang}, you need version $3.2$ or higher.

The tool is platform independent.
  We tested compilation and functionality on
  all three major operating systems, on Linux (Ubuntu 12.04),
  Mac OS X (10.9) and Windows (8.1).

\subsection{Unit testing.}
Unit testing has became a common part of software development process
  in the recent years.
Correctness was a top priority during the development and
  unit tests are a perfect way to capture potential programmer's error
  as soon as possible and avoid regression.

There is a lot of unit tests framework for C++.
We focused on simplicity, minimal amount of work needed to add new tests
  and good assertion support, and opted for
  \emph{Google Test}\ref{google-test}.

All available tests are compiled and excecuted if you run \texttt{make test}
  in the root folder.
This should serve as a basic sanity test and we highly recommend
  doing this in case anyone needs to change something in the code.

\section{Sat solving}


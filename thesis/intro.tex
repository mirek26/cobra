 \chapter{Introduction}


Code breaking games (also known as Deductive games or Searching games)
are games for two players, in which the first, usually reffered to as
\emph{the codemaker}, chooses a secret code from a given set, and the second,
reffered to as \emph{the codebreaker}, strives to reveal the code by a series
of experiments that give him partial information about the code.

The famous board game of Mastermind is a prominent example.
...

Anothe example is the Counterfeit coin problem, the problem of determining
a counterfeit coin among authentic ones using just a balance scale.
Here, the codemaker is not a real player. The balance scale takes his function
and evaluates the experiments -- weighings -- permfomred by the codebreaker.

Numerous other examples can be found among various board games and logic puzzles,
 some of which are presented in \autoref{ch:games}.

% problémy

Although Mastermind and the Conterfeit coin problem have been subjected to
heavy research, few have been written about Code Breaking Games in general.
Some authors suggested general methods (and applied them one of the games),
some vaguely stated that their approach can be applied to other games of this kind but,
to the best of our knowledge, no one have tried to formalize and give some
general results for these games in general.

Here comes this thesis to fill the gap. We develop a formalism
...

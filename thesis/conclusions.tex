\chapter{Conclusions}

We presented a general model of code-breaking games based on propositional logic,
  which can fit Mastermind, the counterfeit coin problem and many others.
Experiment equivalence was introduced and we proved that
  equivalent experiments can be neglected during the analysis of the game.
We suggested an algorithm for equivalence testing based on graph isomorphism.

Further, a computer language for game specification was suggested,
  and we developed a computer tool COBRA for code-breaking game analysis,
  which is able to perform various tasks with the game.

Using the tool, we can reproduce some known results and easily examine
  new strategies.
We provided experimental results and evaluated performances of
  common one-step look-ahead strategies in other games than Mastermind.
We also suggested new strategies based on our model of the game,
  mostly on the number of fixed variables, and analysed their performance.

There are many interesting things to try in our framework,
  which were, however, beyond the scope of the thesis.
In the next paragraphs, we present a few suggestion for future work.

Our model of a code-breaking games is general,
  the are many ways to extend it further.
Experiments price.
Limited number of experiments of a type.


It would be also nice to try other approaches in our framework.
Many were suggested for Mastermind but can be applied in the general case as well.
In particular, genetic algorithms proved to be very useful for bigger problems
  as they scale much better than the backtracking approach.

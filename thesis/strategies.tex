\section{Strategies in general}

This section introduces the concept of a strategy for experiment selection.
We define worst-case and average-case number of experiments of a strategy
 and optimal strategies. Further, we examine several strategy classes.

\begin{definition}[Strategy]\label{def:strategy}
A \emph{strategy} is a function $\stg: \Omega^* -> \Exp$,
  determining the next experiment for a given finite solving process.
\end{definition}

A strategy $\stg$ together with a valuation $\val\in\Vals$
  induce an infinite solving process
  \[
  \procstg{\stg}{\val} = \exp_1, \form_1, \exp_2, \form_2, ...,
  \]
  where
  $\exp_{i+1} = \stg(\procstg{\stg}{\val}[1:i])$
  and
  $\form_{i+1}$ is the formula from $\outcome(\exp_{i+1})$
  satisfied by $\val$,
  for all $i\in\Nset$.
Thanks to the well-formed property,
  there is exactly one such $\form_{i+1}$.

We define \emph{length} of a strategy $\stg$ on a valuation $\val$,
  denoted $\stglen{\stg}{\val}$,
  as the smallest $k\in\Nseto$ such that
  $\stgknow{\stg}{\val}{k}$ uniquely determines the code, i.e.
  \[
  \stglen{\stg}{\val} = \min \;\{ k\in\Nseto \| \numval{\stgknow{\stg}{\val}{k}} = 1 \}
  \]


The \emph{worst-case number of experiments} $\lenmax{\stg}$
  of a strategy $\stg$ is the maximal length of the strategy on a valuation $\val$,
  over all $\val\in\Vals$, i.e.
  \[
  \lenmax{\stg} = \max_{\val\in\Vals} \stglen{\stg}{\val}.
  \]

The \emph{average-case number of experiments} $\lenexp{\stg}$
  of a strategy $\stg$ is the expected number of experiments if the code
  is selected from models of $\init$ with uniform distribution, i.e.
  \[
  \lenexp{\stg} = \frac{
    \sum_{\val\in\Vals} \stglen{\stg}{\val}
    }{\numval{\init}}.
  \]

We say that a strategy $\stg$ \emph{solves the game} if $\lenmax{\stg}$ is finite.
Note that $\lenmax{\stg}$ is finite if and only if $\lenexp{\stg}$ is finite.
The game is \emph{solvable} if there exists a strategy that solves the game.

\medskip

\begin{definition}[Optimal strategy]
A strategy $\stg$ is \emph{worst-case optimal} if
  $\lenmax{\stg} <= \lenmax{\stg'}$ for any strategy $\stg'$.
A strategy $\stg$ is \emph{average-case optimal} if
  $\lenexp{\stg} <= \lenexp{\stg'}$ for any strategy $\stg'$.
\end{definition}

The following lemma provides us with a lower bound on the number of
experiments of a worst-case optimal strategy.

\begin{lemma} \label{lma:lbound}
Let $b = \max_{\expt\in\Expt} |\outcome(\expt)|$ be the maximal number of
  possible outcomes of an experiment. Then for every strategy $\stg$,
  \[
  \lenmax{\stg} >= \lceil \log_b(\numval{\init}) \rceil.
  \]
\end{lemma}

\begin{proof}
Let us fix a strategy $\stg$ and $k = \lenmax{\stg}$.
For an unknown model $\val$ of $\init$,
  $\stgknow{\stg}{\val}{k}$ can take up to
  $b^k$ different values.
By pigeon-hole principle, if $\numval{\init} > b^k$, there must be a valuation
  $v$ such that $\numval{\stgknow{\stg}{\val}{k}} > 1$.
This would be a contradiction with $k = \lenmax{\stg}$ and, therefore,
  $\numval{\init} <= b^k$, which is equivalent with the statement of the lemma.
  \qed
\end{proof}

\begin{lemma} \label{lma:accumulatedknowledge}
Let $\stg$ be a strategy and let $\val_1, \val_2 \in\Vals$.
If $\val_1$ is a model of $\stgknow{\stg}{\val_2}{k}$,
  then $\procstg{\stg}{\val_1}[1:k] = \procstg{\stg}{\val_2}[1:k]$.
\end{lemma}

\begin{proof}
Let $\proc_1 = \procstg{\stg}{\val_1}$, $\proc_2 = \procstg{\stg}{\val_2}$
and consider the first place where $\proc_1$ and $\proc_2$ differs.
It cannot be an experiment $\proc_1(i) \not= \proc_2(i)$ as they are both
  values of the same strategy on the same process:
$\proc_1(i) = \stg(\proc_1[1:i-1]) =
              \stg(\proc_2[1:i-1]) = \proc_2(i)$.

Suppose it is an outcome of the $i$-th experiment, $\proc_1[i] \not= \proc_2[i]$
  and $i <= k$.
Since $\val_1$ satisfies $\aknow{\proc_2}{k}$ and $i <= k$,
  it satisfies $\proc_2[i]$ as well.
However, $\val_1$ always satisfies $\proc_1[i]$ and
  both $\proc_1[i]$ and $\proc_2[i]$ are from the set
  $\outcome(\proc_1(i)) = \outcome(\proc_2(i))$.
Since there is exactly one satisfied experiment for each valuation in the set,
  $\proc_1[i]$ and $\proc_2[i]$ must be the same.
Contradiction. \qed
\end{proof}

\begin{example} \label{ex:run2}
Recall our running example of the counterfeit coin problem with 4 coins,
as defined in \autoref{ex:run1}.

Consider a strategy $\stg$ defined as follows.
For simplicity, we denote experiments by their parametrizations only
  and the outcomes by a symbol $<$, $>$ and $=$,
  instead of the corresponding formula.
\[
\stg(\proc) = \left\{\begin{array}{ll}
13 & \textrm{ if } \proc = (12, <), \\
23 & \textrm{ if } \proc = (12, >),\\
14 & \textrm{ if } \proc = (12, =), (12, =), \\
34 & \textrm{ if } \proc = (12, =), (12, =), (14, =), \\
12 & \textrm{ otherwise.}
\end{array}\right.
\]

Let $\val\in\Vals$ be a valuation such that $v(x_3) = v(y) = 1$.
The induced solving process is
\[
\procstg{\stg}{\val} = (12, =), (12, =), (14, =), (34, >), (12, =), (12, =), ...
\]
The length of $\stg$ on $\val$ is 4, because $v$ is the only model of
  the accumulated knowledge after 4 experiments,
\[
\exactly_1(x_1,x_2,x_3,x_4) \wedge \neg(x_1 \vee x_2) \wedge \neg(x_1 \vee x_2)
\wedge \neg(x_1 \vee x_4) \wedge ((x_3 \wedge y) \vee (x_4 \wedge \neg y)).
\]

The strategy is intentionally inefficient and repeats the experiment $12$
if the outcome in the first step is `$=$'.
In fact, every valuation is discovered by $\stg$ in at most 4 experiments,
  so $\lenmax{\stg} = 4$.

\autoref{lma:lbound} gives us a lower bound $ \lceil\log_3(8)\rceil = 2$
 on the worst-case number of experiments of an optimal strategy.
However, we already know from \autoref{th:coins12} that the minimal number
  of experiments needed to reveal the code is 3.
\end{example}

\subsection{Non-adaptive strategies}

Non-adaptive strategies correspond to the well-studied problems of
  static Mastermind and
  non-adaptive strategies for
  the counterfeit coin problem\cite{mm-static}\cite{coins-nonadaptive}.
We define them here only to show the possibility of formulating the
  corresponding problems
  in our framework but we do not study them any further.

\begin{definition}[Non-adaptive strategy]
A strategy $\stg$ is \emph{non-adaptive} if it decides the next experiment
  based on the length of the solving process only, i.e.
  whenever $\proc_1$ and $\proc_2$ are processes such that
  $|\proc_1| = |\proc_2|$,
  then
  $\stg(\proc_1) = \stg(\proc_2)$.

Non-adaptive strategies can be considered functions $\stgx: \Nseto -> \Exp$,
where $\tau(|\proc|) = \stg(\proc)$.
\end{definition}

\subsection{Memory-less strategies}

The general definition of a strategy allows for the next experiment
  to depend on the exact history of the solving process,
  not only on the accumulated knowledge.
This is in a sense unintuitive, as
  the nature of code-breaking games is memory-less
  and the course of a game depends only on the accumulated knowledge.

\begin{definition}[Memory-less strategy]
A strategy $\stg$ is \emph{memory-less} if it decides the next experiment
  based on the accumulated knowledge only, i.e.
  whenever $\proc_1$ and $\proc_2$ are processes such that if
  $\tknow{\proc_1} \equiv \tknow{\proc_2}$
  then
  $\stg(\proc_1) = \stg(\proc_2)$.

Memory-less strategies can be considered functions
  $\stgx: \Formr -> \Exp$ such that
  $\form_1\equiv\form_2 ==> \stgx(\form_1)=\stgx(\form_2)$.
Then $\stg(\proc) = \stgx(\tknow{\proc})$.
\end{definition}

Note that the number of non-equivalent formulas over variable $\Var$
  is finite and, therefore, the number of memory-less strategies for a fixed
  code-breaking game
  is finite as well.

Now we prove some basic properties of memory-less strategies.
The following lemma says that once we do not get any new information
  from the experiment selected by a experiment,
  we never get any new information with the strategy.
Then, the theorem below proves that there exists an optimal
  memory-less strategy.

\begin{lemma}
Let $\stg$ be a memory-less strategy and $\val\in\Vals$.
If there exists $k\in\Nset$ such that
  $\numval{\stgknow{\stg}{\val}{k}} = \numval{\stgknow{\stg}{\val}{k+1}}$,
 then
  $\numval{\stgknow{\stg}{\val}{k}} = \numval{\stgknow{\stg}{\val}{k+l}}$
 for any $l\in\Nset$.
\end{lemma}

\begin{proof}
To simplify the notation, let $\know^k = \stgknow{\stg}{\val}{k}$.
There is a formula $\form\in\outcome(\know^k)$,
  such that $\know^{k+1} \equiv \know^{k} \wedge \form$.
Therefore, if $\know^{k+1}$ is satisfied by valuation $\val$, so must be $\know^{k}$.
Since $\numval{\know^{k}} = \numval{\know^{k+1}}$, the sets of
  valuations satisfying $\know^{k}$ and $\know^{k+1}$ are exactly the same
  and the formulas are thus equivalent.
This implies $\stg(\know^{k}) = \stg(\know^{k+1})$ and $\know^{k+2} \equiv \know^{k+1}\wedge \form \equiv \know^{k+1}$.

By induction,
  $\stg(\know^{k+l}) = \stg(\know^{k})$ and
  $\know^{k+l} \equiv \know^{k}$
  for any $l\in\Nset$.\qed
\end{proof}

\begin{theorem} \label{th:memless}
Let $\stg$ be a strategy.
Then there exists a memory-less strategy $\stgx$ such that
  $\stglen{\stg}{\val} >= \stglen{\stgx}{\val}$ for all $\val\in\Vals$.
\end{theorem}

\begin{proof}
Let us choose any total order $\form_1, \form_2, ...$ of $\Formr$ such that
  if $\form_i$ implies $\form_j$, then $i <= j$.
We build a sequence of strategies $\stg_0, \stg_1, \stg_2, ...$ inductively in the following way.
Let $\stg_0 = \stg$.
\begin{itemize}
\item If there is no $v\in\Vals, k\in\Nseto$ such that
  $\stgknow{\stg_{i-1}}{v}{k} \equiv \form_i$, select any $\exp\in\Exp$ and
  define $\stg_i$ by
\[
\stg_i(\proc) = \left\{
 \begin{array}{lll}
 \stg_{i-1}(\proc)  & \textrm{ if } \tknow{\proc}\not\equiv\form_i,\\
 \exp               & \textrm{ if } \tknow{\proc}\equiv\form_i.
 \end{array}
 \right.
\]
Clearly, all induced solving processes for $\stg_i$ and $\stg_{i-1}$ are the same
  and $\stglen{\stg_i}{v} = \stglen{\stg_{i-1}}{v}$.

\item If there exists $v\in\Vals, k\in\Nseto$ such that
  $\stgknow{\stg_{i-1}}{v}{k} \equiv \form_i$, choose the largest $l$ such that
  $\stgknow{\stg_{i-1}}{v}{l} \equiv \form_i$ and define
\[
\stg_i(\proc) = \left\{
 \begin{array}{lll}
 \stg_{i-1}(\proc)            & \textrm{ if } \tknow{\proc}\not\equiv\form_i,\\
 \procstg{\stg_{i-1}}{v}(l)   & \textrm{ if } \tknow{\proc}\equiv\form_i.
 \end{array}
 \right.
\]
First we prove that this definition is correct.
Let $v_1, v_2, k_1, k_2$ be such that
  $\stgknow{\stg_{i-1}}{v_1}{k_1}\equiv\form_i\equiv\stgknow{\stg_{i-1}}{v_2}{k_2}$.
Take $l_1, l_2$ as the largest numbers such that
  $\stgknow{\stg_{i-1}}{v_1}{l_1}\equiv\form_i\equiv\stgknow{\stg_{i-1}}{v_2}{l_2}$.
Since $v_1$ satisfies $\stgknow{\stg_{i-1}}{v_2}{l_2}\equiv\form_i$,
  then $\procstg{\stg_{i-1}}{v_2}[1:l_2] = \procstg{\stg_{i-1}}{v_1}[1:l_2]$
  by \autoref{lma:accumulatedknowledge}.
The same holds for $l_1$ which means that $l_1 = l_2$ and
  $\procstg{\stg_{i-1}}{v_1}(l_1) = \procstg{\stg_{i-1}}{v_1}(l_2)$, which
  proves that the definition of $\stg_i$ is independent of the exact choices
  of $v$ and $k$.

Now $\stglen{\stg_i}{v} = \stglen{\stg_{i-1}}{v} - (l-k)$, where
  $k$ and $l$ is the smallest and the largest number such that
  $\stgknow{\stg_{i-1}}{v}{k}\equiv\form_i $ and
  $\stgknow{\stg_{i-1}}{v}{l}\equiv\form_i $, respectively,
  because
  $\procstg{\stg_{i-1}}{v}(l) = \procstg{\stg_{i}}{v}(k)$ and due to the ordering,
  the rest of the process is independent of the beginning.
\end{itemize}

The last strategy of the sequence is clearly memory-less and satisfies the
  condition in the lemma. \qed
\end{proof}

\begin{corollary}
There exists a worst-case optimal strategy that is memory-less
and there exists an average-case optimal strategy that is memory-less.
\end{corollary}

\begin{example}
Recall the game and the strategy $\stg$ from \autoref{ex:run2}.
The strategy is clearly not non-adaptive, as
  $\stg((12, <)) \not= \stg((12, >))$.
It is neither memory-less as
  $\stg((12, =)) \not= \stg((12,=),(12,=))$ but
  the accumulated knowledge of the solving processes is the same.

Consider a non-adaptive strategy
 $\stgx:\; 1 \mapsto 12,\; 2\mapsto 13,\; 3\mapsto 14$.
If the counterfeit coin is among the first three,
  it is discovered by the strategy in two experiments.
If the counterfeit coin is coin 4, it requires three experiments.
Hence $\lenmax{\stgx} = 3$ and the value of $\stgx$ on
  greater numbers is irrelevant.

If we apply the construction in \autoref{th:memless} on $\stg$,
we get a memory-less strategy $\stg'$, given by

\[
\stg'(\form) = \left\{\begin{array}{ll}
13 & \textrm{ if } \form \equiv (x_1 \wedge \neg y) \vee (x_2 \wedge y), \\
23 & \textrm{ if } \form \equiv (x_1 \wedge y) \vee (x_2 \wedge \neg y),\\
14 & \textrm{ if } \form \equiv \neg x_1 \wedge \neg x_2, \\
34 & \textrm{ if } \form \equiv \neg x_1 \wedge \neg x_2 \wedge \neg x_4, \\
12 & \textrm{ otherwise.}
\end{array}\right.
\]

Notice that the valuation $v$ with $v(x_3) = v(y) = 1$ is discovered in
  3 experiments as the strategy does not repeat the experiment 12 now.
Therefore, $\lenmax{\stg'} = 3$.

Both strategies $\tau$ and $\stg'$ are worst-case optimal. \eqed
\end{example}

\section{One-step look-ahead strategies} \label{sec:oslas}

Specification of a strategy in general can be very complicated.
In this section, we study a subclass of memory-less strategies that we call
  \emph{one-step look-ahead}.
These strategies select an experiment that
  minimizes the value of a given function
  on the set of possible knowledge in the next step.

\newcommand{\formset}{\Psi}
\begin{definition}[One-step look-ahead strategy]\label{def:oslas}
Let $f$ be a function of type $2^{\Formr} -> \Rset$ and
  $\eord$ a total order of experiments.
A one-step look-ahead strategy with respect to $f$ and $\eord$ and is
  a memory-less strategy $\stg$, such that
  $\stg(\form) = e$ is the minimal element of $(E,\eord)$ satisfying
\[
\forall e'\in\Exp. \;\; f(\{\:\form \wedge \formx \| \formx\in\outcome(e) \:\}) <=
  f(\{\: \form \wedge \formx \| \formx\in\outcome(e') \:\}).
\]
\end{definition}

We refer to the value $f(\{\form \wedge \formx \| \formx\in\outcome(e)\})$
  as the value of the function $f$ on experiment $e$.
The purpose of $\eord$ is to resolve the cases in which
  the value of the function is the same for more experiments.

% Note that one-step look-ahead strategy with respect to $f$ is not unique.
% For some formulas, there can be more experiments with the same value of $f$.
% To uniquely specify a strategy, we must provide the function $f$ and
%   a resolution method for these ambiguous states.
% Typically, we specify a total order on experiments and select the least
%   experiment in the order satisfying the condition of \autoref{def:oslas}.

A few one-step look-ahead strategies for Mastermind
  have been already introduced in \autoref{sec:mm}.
In the following, we define them formally for the general model of code-breaking games.
Unless otherwise stated, the total order $\eord$ is
  the lexicographical order of the experiments.

\begin{description}
\item[Max-models.]
This strategy minimizes the worst-case number of remaining codes.
For Mastermind, this was suggested by Knuth\cite{mm-knuth}.
\[
f(\formset) = \max_{\form\in\formset} \numval{\form}.
\]

\item[Exp-models.]
This strategy minimizes the expected number of remaining codes.
  For Mastermind, this was suggested by Irwing\cite{mm-expnum}.
\[
f(\formset) = \frac{\sum_{\form\in\formset}(\numval{\form})^2}{\sum_{\form\in\formset} \numval{\form}}.
\]

\item[Ent-models.]
This strategy maximizes the entropy of the numbers of remaining codes,
For Mastermind, this was suggested by Neuwirth\cite{mm-entropy}.
\[
f(\formset) = \sum_{\form\in\formset} \frac{\numval{\form}}{N} \cdot \log \frac{\numval{\form}}{N}
  \textrm{, where } N = \sum_{\form\in\formset}\numval{\form}.
\]

\item[Parts.]
This strategy maximizes the number of satisfiable outcomes.
For Mastermind, this was suggested by Kooi\cite{mm-mostparts}.
\[
f(\formset) = - \:|\{ \form \| \form\in\formset, \;\SAT{\form} \}|.
\]
\end{description}

\newcommand{\fixed}{\#_\textrm{fixed}\:}
We suggest and analyse two more one-step look ahead strategies based on
the number of fixed variables of the formula.
Let
\[
\fixed\form = |\{ x\in\Var \| \forall v.v(\form)=1 ==> v(x)=1 \}
                  \cup\{ x\in\Var \| \forall v.v(\form)=1 ==> v(x)=0 \}|
\]
be the number of variables of $\form$.
Note that while the aforementioned strategies does not depend on the exact
  formalization of a problem, the number of fixed variables may differ for
  different encodings.
For example, the choice of the following strategies in \autoref{ex:cc1} differs
  for the two possible formalisations.

\begin{description}
\item[Min-fixed.] Maximize the worst-case number of fixed variables, i.e.
\[
f(\formset) = -\min_{\form\in\formset} \fixed{\form}.
\]
\item[Exp-fixed.] Maximize the expected number of fixed variables, i.e.
\[
f(\formset) = -\frac{\sum_{\form\in\formset}\numval{\form}\cdot\fixed{\form}}{\sum_{\form\in\formset}\numval{\form}}.
\]
\end{description}
\pagebreak

\begin{example}
Recall \autoref{ex:run1} and consider two experiments in the first step.
First, consider an experiment of weighing coin 1 against coin 2.
All the 3 outcomes are satisfiable, the number of models is
  2, 2 and 4 for outcome $<$, $>$, and $=$, respectively.
If the experiment results in $<$ or $>$, we know that the counterfeit coin
is coin 1 or coin 2. If it results in $=$, the counterfeit coin is coin 3 or coin 4.
Therefore, every outcome fixes two variables.

Second, consider an experiment of weighing coins 1 and 2 agains coins 3 and 4.
As exactly one coin must be counterfeit, outcome $=$ is not possible.
Outcomes $<$ and $>$ are symmetrical, both have $4$ models and fix no variables.

\begin{table}[!ht]
\begin{center}
\begin{tabular}{l|cc}
& 12 & 1234 \\\hline
Max-models & 4 & 4 \\
Exp-models & 3 & 4 \\
Ent-models & -1.04 & -0.69 \\
Parts & -3 & -2 \\
Min-fixed & -2 & 0 \\
Exp-fixed & -2 & 0 \\
\end{tabular}
\caption{Values of various one-step look-ahead strategies in the counterfeit \\
  coin problem with four coins on experiments 12 and 1234.}
\label{tbl:run-oslas}
\end{center}
\end{table}

\autoref{tbl:run-oslas} shows the values of the defined
  strategies on these two experiments.
The experiment $12$ wins with all strategies except for Maximal number of models,
  where the values are the same. \eqed
\end{example}

% Greedy strategies are optimal in the fake-coin game $\mathcal{F}_n$.

% \TODO{Napsat důkaz.}



\section{Symmetries in Code Braking Games}

The number of possible parametrizations of a type of experiment is typically very large,
  which makes the analysis a game much harder.
For example, consider the conterfeit-coin problem (\ref{prob-coins})
  and the experiment of weighting 4 coins against 4 coins.
There is $\frac{1}{2}\cdot {12 \choose 4}\cdot{8 \choose 4} = 17325$
 possible combinations for parametrization, but
 in the initial state (i.e. with no knowledge except for the initial restriction),
 all of them are equivelent -- they will give us the same information regardless of symmetries.

In this section, we formally define equivalence of two experiments
 and show that we can neglect all but one experiment in each equivelence class.
Further, we present several lemmas that will form the basis of
 our symmetry breaking algorithms in the Chapter \ref{chapter-cobra}.

\begin{definition}[Symmetric experiment]
For an experiment $\exp = (\expt, \param)$ and a permutation $\perm\in\Perm_\Var$,
  a $\perm$-symmetric experiment $\exp^\perm = (\expt, \param')\in\Exp$
  is an experiment of the same type such that
  $\{\form^\perm \in\outcome(\exp)\} = \{\form\in\outcome(\exp^\perm)\}$.
Clearly, no such experiment may exists.
\end{definition}

\begin{definition}[Symmetry group]
We define a \emph{symmetry group} $\symg$ as
  the maximal subset of $\Perm_\Var$ such that for
  every $\perm\in\symg$ and for every experiment $\exp\in\Exp$,
  there exists a $\perm$-symmetric experiment $\exp^\perm$.
\end{definition}

\begin{definition}[Consistent strategy]
A memory-less strategy $\stg$ is \emph{consistent} if and only if
  for every $\form\in\Form_\Var$ and every $\perm\in\symg$, there
  exists $\permx\in\symg$ such that $\form^\perm \equiv \form^\permx$ and
  $\stg(\form^\permx) = \stg(\form)^\permx$.
\end{definition}

\TODO{Example, na kterém bude vidět, že jednoduchá definice nevyhovuje, protože můžu vzít symetrie $\form$ a dostanu, že to má dávat různé věci.}

\begin{lemma}
Let $\stg$ be a memory-less strategy.
There exists a consistent memory-less strategy $\stgx$ such that
  $\stglen{\stg}{\val} >= \stglen{\stgx}{\val}$
  for all $\val\in\Val_X$ satisfying $\init$.
\end{lemma}

\begin{proof}
Similar to the proof of Lemma \ref{lma-opt-memoryless}.
Let us choose any total order $\form_1, \form_2, ...$ of $\Form_\Var$ such that
if $\form_i$ implies $\form_j$, then $i <= j$.
We build a sequence of strategies $\stg_0, \stg_1, \stg_2, ...$ in the following way:
Let $\stg_0 = \stg$ and for $i > 0,$
\begin{equation}
\stg_i(\form) = \left\{
 \begin{array}{lll}
 \stg_{i-1}(\form) & \textrm{ if } \not\exists\perm\in\symg.\form^\perm\equiv\form_i \\
 \procstg{\stg_{i-1}}{\val_i}(k_i + 1) &
    \textrm{ if } \exists\perm\in\symg.\form^\perm\equiv\form_i\textrm{, where }
    (v_i, k_i) = \argmin_K \stglen{\stg_{i-1}}{v} - k,\\
    & \textrm{ and }
    K = \{(v, k)\in\Val\times\Nseto \| \exists\perm.\stgknow{\stg_{i-1}}{v}{k}^\perm\equiv\form_i\}.
    \vspace{-2.5ex}
 \end{array}
 \right.
 \vspace{2.5ex}
\end{equation}
We prove that $\stglen{\stg_{i}}{v} <= \stglen{\stg_{i-1}}{v}$.
If there is no $k$ and $\perm$ such that $\stgknow{\stg_i}{v}{k}^\perm\equiv\form_i$ then
  the processes $\procstg{\stg_i}{v}$ and $\procstg{\stg_{i-1}}{v}$ are the same.
If the is such $k$ and \TODO{...}.

The last strategy of the sequence is consistent and satisfies the
  condition in the lemma. \qed
\end{proof}

\begin{definition}[Experiment equivalence]
An experiment $\exp_1\in\Exp$ is equivalent to $\exp_2\in\Exp$ with respect to $\form$,
  written $\exp_1\expeq{\form}\exp_2$,
  if and only if there exists a permutation $\perm\in\symg$ such that
 $ \{ \form\wedge\formx \| \formx\in\outcome(\exp_1) \} \equiv
   \{ (\form\wedge\formx)^\perm \| \formx\in\outcome(\exp_2) \} $.
\end{definition}

\begin{theorem}
Let $\stg, \stgx$ be two consistent memory-less strategies, such that
$\stg(\form) \expeq{\form} \stgx(\form)$ for any $\form\in\Form_\Var$.
There is a bijection $f:\Val_\Var -> \Val_\Var$ such that
$\stglen{\stg}{\val} = \stglen{\stgx}{f(\val)}$.
\end{theorem}

\begin{proof}
First, we prove by induction for any $k\in\Nseto$,
  there is a permutation $\perm\in \symg$ such that
  $(\stgknow{\stg}{\val}{i})^\perm = \stgknow{\stgx}{\val^\perm}{i}$
  for all $i\in\Nseto$, $i<=k$.
For better readability, let
  $\know_k = \stgknow{\stg}{\val}{k}$ and
  $\knowx_{k, \perm} = \stgknow{\stgx}{\val^\perm}{k}$

For $k=0$, take $\perm = \idperm_\Var$.
Clearly, $\stgknow{\stg}{\val}{0} = \init = \stgknow{\stgx}{\val^\idperm}{0}$.

For the induction step, suppose we have $\perm\in\symg$ such that
  $\know_i^\perm = \knowx_{i, \perm}$ for $i <= k$.
Further, suppose $\perm$ is such that
  $\stg(\know_k^\perm) = \stg(\know_k)^\perm$. \TODO{Víc zdůvodnit.}

Let $e_1 = \stg(\know_k)$, $e_2 = \stgx(\knowx_{k, \perm})$
  be the $(k+1)$-th experiments  of the strategies.
It holds
\begin{equation}
e_2 = \stgx(\knowx_{k, \perm})
    \expeq{\knowx_{k, \perm}}  \stg(\knowx_{k, \perm})
    \stackrel{IH}{=} \stg(\know_k^\perm)
    = \stg(\know_k)^\perm
    = e_1^\perm \label{eq:expsym}\tag{$\sim$}
\end{equation}

and, therefore, there exists $\permx\in\symg$ such that
\begin{align}
 \{ \knowx_{k, \perm} \wedge \formx \| \formx\in\outcome(\exp_2) \} &= %\stackrel{(\refeq{eq:expsym})}{=}
 \{ (\knowx_{k, \perm} \wedge \formx)^\permx \| \formx\in\outcome(\exp_1^\perm) \} = \label{eq:sets}\tag{*}\\
&= \{ (\know_k^\perm \wedge \formx^\perm)^\permx \| \formx\in\outcome(\exp_1) \} =
 \{ (\know_k \wedge \formx)^{\permx\perm} \| \formx\in\outcome(\exp_1) \}
\end{align}
As $\permx\in\symg$ and $\symg$ is a permutation group, $\permx\perm\in\symg$.

Since the game is well-formed,
  $v$ satisfies exactly one formula in
  $\{ \know_k \wedge \formx \| \formx\in\outcome(\exp_1) \}$.
Therefore $v^{\permx\perm}$ satisfies exactly one formula
  in
  $\{ (\know_k \wedge \formx)^{\permx\perm} \| \formx\in\outcome(\exp_1) \}  =
   \{ \knowx_{k, \perm} \wedge \formx \| \formx\in\outcome(\exp_2) \}$,
  which means that $v^{\permx\perm}$ satisfies $\knowx_{k, \perm}$.
\TODO{Tohle nefugujeee!}
From Lemma \ref{lma-accruedknowledge},
  $\knowx_{k, \perm} = \knowx_{k, \permx\perm}$.
Both $\know_{k+1}^{\permx\perm}$ and $\knowx_{k+1, \permx\perm}$ is thus the only
  formula from (\refeq{eq:sets}) satisfied by $v^{\permx\perm}$ and, therefore,
  $\know_{k+1}^{\permx\perm} = \knowx_{k+1, \permx\perm}$.



Now for a fixed $\val$, take $k = \stglen{\stg}{\val}$, take
  $\perm\in\symg$ such that
  $(\stgknow{\stg}{\val}{k})^\perm = \stgknow{\stgx}{\val^\perm}{k}$
  and define $f(\val) = \val^\perm$.
Since
 $(\stgknow{\stg}{\val}{i})^\perm = \stgknow{\stgx}{f(\val)}{i}$
 for $i <= k$
 and variable permutation preserves the number of models of a formula, i.e.
  $\numval{\form} = \numval{\form^\perm}$ for any
  $\form\in\Form_\Var$, $\perm\in\Perm_\Var$,
 we have
  $\stglen{\stg}{\val} = \stglen{\stgx}{f(\val)}$.
% \TODO{It remains to show that $f$ is a bijection.}
  % Suppose $f$ is not injective, and $f(\val_1) = f(\val_2)$.
  % By definition, the only model of
  \qed
\end{proof}

\begin{corollary}
Let $\stg_1, \stg_2$ be two consistent memory-less strategies, such that
  $\stg_1(\form) \expeq{\form} \stg_2(\form)$ for any $\form\in\Form_\Var$.
Then $\lenmax{\stg_1} = \lenmax{\stg_2}$
  and $\lenexp{\stg_1} = \lenexp{\stg_2}$.
\end{corollary}

For any accrued knowledge $\form$, this lemma gives us the right
to consider only one of the experiments
$\exp_1, \exp_2$ if $\exp_1 \sim_\form \exp_2$.

Now, we would love an algorithm that would, for a given formula $\form$,
generate a set of experiment, such that there is exaclty one expriment
from every equivalence class in $E/\sim_\form$.

% \TODO{...}

% For the following sections, let us fix an experiment type $t$.
% \subsection{Phase 1}

% \subsection{Phase 2}
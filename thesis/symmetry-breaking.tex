\section{Symmetry Breaking}

% The number of possible parametrizations of a type of experiment is typically very large,
%   which makes the analysis a game much harder.
% For example, consider the conterfeit-coin problem (\ref{prob-coins})
%   and the experiment of weighting 4 coins against 4 coins.
% There is $\frac{1}{2}\cdot {12 \choose 4}\cdot{8 \choose 4} = 17325$
%  possible combinations for parametrization, but
%  in the initial state (i.e. with no knowledge except for the initial restriction),
%  all of them are equivelent -- they will give us the same information regardless of symmetries.

% In this section, we formally define equivalence of two experiments
%  and show that we can neglect all but one experiment in each equivelence class.
% Further, we present several lemmas that will form the basis of
%  our symmetry breaking algorithms in the Chapter \ref{chapter-cobra}.

%Since composition of permutations is a permutation, this is clearly an equivalence relation.
% \TODO{Tohle hrozně nefunguje. Musí být symetrické i experimenty.}
% Plán:
% \begin{itemize}
% \item zadefinovat symetrii množiny experimentů jako gropu permutací proměnných tak, aby $\forall e\in E \exists e^\pi$ (kde
% ).
% \item zadefinovat nějakou vlastnost hry (symetrická) - pokud pro nějaké $e$ existuje $e^\pi$, pak pro všechny
% \item říct, že pokud jsou parametrizace nějak normálně omezené (typicky libovolný řetězec, nebo řetězec bez opakování), tak je ta hra symetrická
% \item omezit se s analýzou jen na symetrické hry
% \item zadefinovat konzistentní strategii, jako strategii, která na zpermutovanou formuli zahraje zpermutovaný experiment
% \item zadefinovat ekvivalenci experimentů podle definice výše
% \item dokázat, že pokud mám strategie $\stg_1. \stg_2$ tak že $\stg_1(\form) \sim_\form \stg_2(\form)$,
% tak existuje permutace proměnných taková, že $|\procstg{\stg_1}{\val}| = |\procstg{\stg_2}{\val^\pi}|$
% \end{itemize}

\begin{definition}
For an experiment $\exp$ and a permutation $\perm\in\Perm_\Var$,
  a $\perm$-symmetric experiment $\exp^\perm\in\Exp$ is an experiment such that
  $\{\form^\perm \in\outcome(\exp)\} = \{\form\in\outcome(\exp^\perm)\}$.
Clearly, no experiment satisfying this may exists.
\end{definition}

\begin{definition}
A game $\game$ is \emph{symmetric} if it satifies the following implication:
If for an experiment $\exp$ exists a $\perm$-symmetric experiment $\exp^\perm\in\Exp$,
  then a $\perm$-symmetric experiment exists for every experiment.
\end{definition}

For the rest of this chapter, the game we analyze is \emph{symmetric}.

\begin{definition}
A memory-less strategy $\stg$ is \emph{consistent} if and only if
  $\stg(\form^\perm) = \stg(\form)^\perm$ for any $\form\in\Form_\Var$ and
  $\perm\in$ symmetry group of the set of experiments.
\end{definition}

\TODO{Aby mohla být strategie konzistentní, musí být hra symetrická.}

\newcommand{\expeq}[1]{\cong_{#1}}
\begin{definition}
An experiment $\exp_1\in\Exp$ is equivalent to $\exp_2\in\Exp$ with respect to $\form$,
  written $\exp_1\expeq{\form}\exp_2$,
  if and only if there exists a permutation $\perm\in\Perm_\Var$ such that
 $ \{ \form\wedge\formx \| \formx\in\outcome(\exp_1) \} \equiv
   \{ (\form\wedge\formx)^\perm \| \formx\in\outcome(\exp_2) \} $.
\end{definition}

\begin{lemma} \label{lma-accruedknowledge}
Let $\stg$ be a consistent memory-less strategy and let $\val_1$, $\val_2$ be two models of $\init$.
If $v_1$ is a model of $\stgknow{\stg}{\val_2}{k}$, then $\stgknow{\stg}{\val_1}{k} = \stgknow{\stg}{\val_2}{k}$.
\end{lemma}
\begin{proof}
Let $\proc_1 = \procstg{\stg}{\val_1}$, $\proc_2 = \procstg{\stg}{\val_2}$
and let $c <= k$ be the first place where $\proc_1$ and $\proc_2$ differs.
...
\end{proof}

\begin{theorem}
Let $\stg_1, \stg_2$ be two consistent memory-less strategies, such that
$\stg_1(\form) \expeq{\form} \stg_2(\form)$ for any $\form\in\Form_\Var$.
Then for any $k\in\Nseto$, there is a permutation $\perm\in Sym$ such that
$\perm\in Sym$ and $(\stgknow{\stg}{\val}{k})^\perm = \stgknow{\stgx}{\val^\perm}{k}$.
\end{theorem}

\begin{proof}
By Induction.

For simplicity, let
  $\know_k = \stgknow{\stg}{\val}{k}$ and
  $\knowx_{k, \perm} = \stgknow{\stgx}{\val^\perm}{k}$
For $k=0$ it clearly holds, as any \TODO{$\perm\in$ Sym(E) $\init = \init^\perm$.}

For the induction step, suppose $\know_k^\perm = \knowx_{k, \perm}$ and $\perm\in Sym$.
Let $e_1 = \stg(\know_k)$, $e_2 = \stgx(\knowx_{k, \perm})$. It holds
\[
e_2 = \stgx(\knowx_{k, \perm})
    \expeq{\knowx_{k, \perm}}   \stg(\knowx_{k, \perm})
    = \stg(\know_k^\perm)
    = \stg(\know_k)^\perm
    = e_1^\perm
\]

and, therefore, there exists $\permx\in Sym$ such that
\begin{align}
 \{ \knowx_{k, \perm} \wedge \formx \| \formx\in\outcome(\exp_2) \} &=
 \{ (\knowx_{k, \perm} \wedge \formx)^\permx \| \formx\in\outcome(\exp_1^\perm) \} = \label{eq:sets}\tag{*}\\
&= \{ (\know_k^\perm \wedge \formx^\perm)^\permx \| \formx\in\outcome(\exp_1) \} =
 \{ (\know_k \wedge \formx)^{\perm\permx} \| \formx\in\outcome(\exp_1) \}
\end{align}
As $\permx\in Sym(E)$ and $Sym$ is a permutation group, $\perm\permx\in Sym(E)$.

Since we suppose the game is well-formed (Definition \ref{def-wellformed}),
  $v$ satisfies exactly one formula in
  $\{ \know_k \wedge \formx \| \formx\in\outcome(\exp_1) \}$.
Therefore $v^{\perm\permx}$ satisfies exactly one formula
  in
  $\{ (\know_k \wedge \formx)^{\perm\permx} \| \formx\in\outcome(\exp_1) \}  =
   \{ \knowx_{k, \perm} \wedge \formx \| \formx\in\outcome(\exp_2) \}$,
  which means that $v^{\perm\permx}$ satisfies $\knowx_{k, \perm}$.
From Lemma \ref{lma-accruedknowledge}, $\knowx_{k, \perm} = \knowx_{k, \perm\permx}$.
Both $\know_{k+1}^{\perm\permx}$ and $\knowx_{k+1, \perm\permx}$ is thus the only
  formula from (\refeq{eq:sets}) satisfied by $v^{\perm\permx}$. \qed
\end{proof}

\begin{corollary}
Let $\stg_1, \stg_2$ be two consistent memory-less strategies, such that
$\stg_1(\form) \expeq{\form} \stg_2(\form)$ for any $\form\in\Form_\Var$.
Then $\lenmax{\stg_1} = \lenmax{\stg_2}$.
\end{corollary}

% For any accrued knowledge $\form$, this lemma gives us the right
% to consider only one of the experiments
% $\exp_1, \exp_2$ if $\exp_1 \sim_\form \exp_2$.

% Now, we would love an algorithm that would, for a given formula $\form$,
% generate a set of experiment, such that there is exaclty one expriment
% from every equivalence class in $E/\sim_\form$.
% \TODO{...}

% For the following sections, let us fix an experiment type $t$.
% \subsection{Phase 1}

% \subsection{Phase 2}
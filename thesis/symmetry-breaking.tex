\section{Symmetry Breaking}

% The number of possible parametrizations of a type of experiment is typically very large,
%   which makes the analysis a game much harder.
% For example, consider the conterfeit-coin problem (\ref{prob-coins})
%   and the experiment of weighting 4 coins against 4 coins.
% There is $\frac{1}{2}\cdot {12 \choose 4}\cdot{8 \choose 4} = 17325$
%  possible combinations for parametrization, but
%  in the initial state (i.e. with no knowledge except for the initial restriction),
%  all of them are equivelent -- they will give us the same information regardless of symmetries.

% In this section, we formally define equivalence of two experiments
%  and show that we can neglect all but one experiment in each equivelence class.
% Further, we present several lemmas that will form the basis of
%  our symmetry breaking algorithms in the Chapter \ref{chapter-cobra}.

%Since composition of permutations is a permutation, this is clearly an equivalence relation.
\TODO{Tohle hrozně nefunguje. Musí být symetrické i experimenty.}
Plán:
\begin{itemize}
\item zadefinovat symetrii množiny experimentů jako gropu permutací proměnných tak, aby $\forall e\in E \exists e^\pi$ (kde
).
\item zadefinovat nějakou vlastnost hry (symetrická) - pokud pro nějaké $e$ existuje $e^\pi$, pak pro všechny
\item říct, že pokud jsou parametrizace nějak normálně omezené (typicky libovolný řetězec, nebo řetězec bez opakování), tak je ta hra symetrická
\item omezit se s analýzou jen na symetrické hry
\item zadefinovat konzistentní strategii, jako strategii, která na zpermutovanou formuli zahraje zpermutovaný experiment
\item zadefinovat ekvivalenci experimentů podle definice výše
\item dokázat, že pokud mám strategie $\stg_1. \stg_2$ tak že $\stg_1(\form) \sim_\form \stg_2(\form)$,
tak existuje permutace proměnných taková, že $|\procstg{\stg_1}{\val}| = |\procstg{\stg_2}{\val^\pi}|$
\end{itemize}

\begin{definition}
For an experiment $\exp$ and a permutation $\perm\in\Perm_\Var$,
  a $\perm$-symmetric experiment $\exp^\perm\in\Exp$ is an experiment such that
  $\{\form^\perm \in\outcome(\exp)\} = \{\form\in\outcome(\exp^\perm)\}$.
Clearly, no experiment satisfying this may exists.
\end{definition}

\begin{definition}
A game $\game$ is \emph{symmetric} if it satifies the following implication:
If for an experiment $\exp$ exists a $\perm$-symmetric experiment $\exp^\perm\in\Exp$,
  then a $\perm$-symmetric experiment exists for every experiment.
\end{definition}

For the rest of this chapter, the game we analyze is \emph{symmetric}.

\begin{definition}
A memory-less strategy $\stg$ is \emph{consistent} if and only if
  $\stg(\form^\perm) = \stg(\form)^\perm$ for any $\form\in\Form_\Var$ and
  $\perm\in$ symmetry group of the set of experiments.
\end{definition}

\begin{definition}
An experiment $\exp_1\in\Exp$ is equivalent to $\exp_2\in\Exp$ with respect to $\form$,
  written $\exp_1\sim_\form\exp_2$,
  if and only if there exists a permutation $\perm\in\Perm_\Var$ such that
 $ \{ \form\wedge\formu \| \formu\in\outcome(\exp_1) \} \equiv
   \{ (\form\wedge\formu)^\perm \| \formu\in\outcome(\exp_2) \} $.
\end{definition}

\begin{theorem}
Let $\stg_1, \stg_2$ be two consistent strategies, such that
$\stg_1(\form) \sim_\form \stg_2(\form)$ for any $\form\in\Form_\Var$.
Then ... ?!? %$\lenmax{\stg_1} = \lenmax{\stg_2}$ ?!?
\end{theorem}

\begin{proof}
\newcommand{\knA}{\varphi}
\newcommand{\knB}{\chi}
Značme $\knA_k, \knB_{k,\perm}$ accumulated knowledge of $\stg, \tau$ after $k$ steps on valuation $v, v^\perm$ respectively.

Indukcí ukážu, že pro $k\in\Nseto$ existuje $\perm\in Sym(E)$, $\knA_k^\perm = \knB_{k,\perm}$.
Pro $k=0$ zřejmě platí. \TODO{Sym(E) vs symetrie $\init$..}

Předpokl. $\knA_k^\perm = \knB_{k, \perm}$.
Oznečme $e_1 = \stg(\knA_k)$, $e_2 = \tau(\knB_{k, \perm})$.

\[
e_2 = \tau(\knB_{k, \perm})
    \sim_{\knB_{k, \perm}}   \stg(\knB_{k, \perm})
    = \stg(\knA_k^\perm)
    = \stg(\knA_k)^\perm
    = e_1^\perm
\]

Díky tomu existuje $\rho\in Sym(E)$ such that
\begin{align}
 \{ \knB_{k, \perm} \wedge \formu \| \formu\in\outcome(\exp_2) \} &=
 \{ (\knB_{k, \perm} \wedge \formu)^\rho \| \formu\in\outcome(\exp_1^\perm) \} = \\
&= \{ (\knA_k^\perm \wedge \formu^\perm)^\rho \| \formu\in\outcome(\exp_1) \} =
 \{ (\knA_k \wedge \formu)^{\pi\rho} \| \formu\in\outcome(\exp_1) \} \\
\end{align}

$\rho\in Sym(E)$takže i $\perm\rho\in Sym(E)$. Ukážeme, že
$\knA_{k+1}^{\pi\rho} = \knB_{k+1, \pi\rho}$.



\end{proof}

% For any accrued knowledge $\form$, this lemma gives us the right
% to consider only one of the experiments
% $\exp_1, \exp_2$ if $\exp_1 \sim_\form \exp_2$.

% Now, we would love an algorithm that would, for a given formula $\form$,
% generate a set of experiment, such that there is exaclty one expriment
% from every equivalence class in $E/\sim_\form$.
% \TODO{...}

% For the following sections, let us fix an experiment type $t$.
% \subsection{Phase 1}

% \subsection{Phase 2}
\documentclass[12pt, final, twoside]{fithesis2}
\usepackage[inner=38mm, outer=27mm, top=37mm, bottom=48mm]{geometry}
\usepackage[utf8]{inputenc}

% searchable and copyable pdf (special characters, accents)
\usepackage{cmap}

% colours and graphics
\usepackage[usenames,dvipsnames,svgnames]{xcolor}

%\usepackage{fontspec}
\usepackage[normalem]{ulem}

\usepackage{subfig}
\usepackage[us,24hr]{datetime}
\newdateformat{mydate}{\THEDAY.\THEMONTH.\THEYEAR}
\newcommand{\verze}{  \scriptsize(v. \mydate\today\ \currenttime)}

% font
\usepackage{lmodern}
\usepackage{microtype}
\usepackage{indentfirst}
% \usepackage{verbatim}
% \usepackage{moreverb}
\usepackage{setspace} % spacing environment

\usepackage{enumitem}
\setitemize{itemsep=.5ex, topsep=.5ex, parsep=0pt, partopsep=0pt}

\usepackage[ruled,lined]{algorithm2e}
\SetAlgorithmName{Algorithm}{Algorithm}

\setlength{\algomargin}{0.5cm}

\usepackage{listings}
\lstset{
  basicstyle=\footnotesize\ttfamily,
  commentstyle=\color{Gray},
  showspaces=false,
  stringstyle=\color{orange},
  %frame=single,
  showstringspaces=false,
  keywordstyle=\bfseries\color{BrickRed},
  commentstyle=\itshape\color{Gray}
}


\newcommand{\argmax}{\operatornamewithlimits{arg\,max}}
\newcommand{\argmin}{\operatornamewithlimits{arg\,min}}

\usepackage[final]{hyperref}
\providecommand*{\problemautorefname}{Problem}
\providecommand*{\exampleautorefname}{Example}
\providecommand*{\lemmaautorefname}{Lemma}
\providecommand*{\definitionautorefname}{Definition}
\providecommand*{\algorithmautorefname}{Algorithm}
\providecommand*{\algocfautorefname}{Algorithm}
%\def\exampleautorefname{Example}
\def\sectionautorefname{Section}
\def\chapterautorefname{Chapter}

% theorems

\usepackage{framed}
\usepackage{wrapfig}

\usepackage{amsmath}
\usepackage{mathtools}  % (in texlive-latex3 package)
\mathtoolsset{showonlyrefs}

%\usepackage[amsmath, amsthm, framed, thmmarks]{ntheorem}
\usepackage[amsmath, amsthm, framed]{ntheorem}

% \usepackage{amsthm}
\usepackage{tikz}
\tikzstyle{thmbox} = [rectangle, rounded corners=10, draw=gray!15, fill=gray!15, inner sep=8pt, inner ysep=1pt]
\newcommand\thmbox[1]{
  % \hspace{-8pt}
  \begin{tikzpicture}
  \node [inner sep=-10pt, inner ysep=-5pt] (box) {
    \begin{tikzpicture}
    \node [thmbox] (box){
      #1
    };
    \end{tikzpicture}
  };
  \end{tikzpicture}
}
\let\theoremframecommand\thmbox

\newcounter{common}

\usepackage{aliascnt}
\newaliascnt{theorem}{common}
\newaliascnt{definition}{common}
\newaliascnt{example}{common}
\newaliascnt{lemma}{common}
\newaliascnt{problem}{common}
\newaliascnt{corollary}{common}

\makeatletter
\renewcommand*{\thealgocf}{\arabic{chapter}.\arabic{algocf}}
\renewcommand*{\thetable}{\arabic{chapter}.\arabic{table}}
\renewcommand*{\thefigure}{\arabic{chapter}.\arabic{figure}}
\let\c@algocf\c@common
\let\c@table\c@common
\let\c@figure\c@common
\makeatother

\theoremstyle{definition}
\newshadedtheorem{definition}{Definition}[chapter]
\newtheorem{example}{Example}[chapter]
\newcommand\xqed[1]{%
  \leavevmode\unskip\penalty9999 \hbox{}\nobreak\hfill
  \quad\hbox{#1}}
\newcommand\eqed{\xqed{$\blacklozenge$}}

\theoremstyle{plain}
\newtheorem{problem}{Problem}[chapter]
\newshadedtheorem{theorem}{Theorem}[chapter]
\newtheorem{lemma}{Lemma}[chapter]
\newtheorem{corollary}{Corollary}[chapter]

\usepackage[justification=centering]{caption}
\usepackage{graphicx}

% cool symbols
\usepackage{amssymb}
\usepackage{MnSymbol}
\usepackage{wasysym}
% \usepackage{marvosym}

\usepackage{csquotes}

\usepackage[style=numeric, sorting=none]{biblatex}

% zkratky pro zprehledneni
\usepackage[ligature]{semantic}
\mathlig{<-}{\leftarrow}
\mathlig{->}{\rightarrow}
\mathlig{<->}{\leftrightarrow}
\mathlig{==>}{\Rightarrow}
\mathlig{<==}{\Leftarrow}
\mathlig{<=}{\leq}
\mathlig{>=}{\geq}
\mathlig{...}{\dotso}

\pagestyle{plain}
\sloppy

%\renewcommand{\chaptermark}[1]{\markright{\thechapter.\ \verze#1}{}}
% \renewcommand{\chaptermark}[1]{ \markboth{#1}{} }
\renewcommand{\sectionmark}[1]{\markright{{\verze}\ \thesection\ #1}{}}

\setlength{\parskip}{.3ex plus .2ex minus .1ex}
\setlength{\parindent}{0pt}
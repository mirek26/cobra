\chapter{Experiment equivalence and algorithms} \label{ch:expeq}

What makes the analysis of code-breaking games difficult is
  typically the large number of experiments.
For example, during the evaluation a one-step look-ahead strategy
  with respect to function $f$,
  we need to compute the value of $f$ on all experiments.
The number of experiments is even more important for optimal strategy synthesis,
  where we have to consider all possible experiments in every state
  and analyse whether the experiment can lead to an improvement
  of the number of experiments of a strategy.

Fortunately, some experiments are usually equivalent to some others in the sense
  that the knowledge they can give us is either exactly the same or symmetrical.
In the counterfeit coin problem, for example,
  the parametrized experiment of weighing 4 coins against 4 coins
  has $\frac{1}{2}\cdot {12 \choose 4}\cdot{8 \choose 4} = 17,325$
  possible parametrizations.
In the initial state, however,
  all of them are equivalent
  as they give us symmetrical knowledge.

%Algorithms for symmetry detection in Mastermind
%   based on graph isomorphism have been suggested in \cite{cbg-nauty}.

This chapter formally introduces the concept of experiment equivalence.
We prove that in various situations, it is sufficient to consider
  one experiment from each equivalence class.
This fact is used in the presented algorithms for well-formed check,
  evaluation of a one-step look-ahead strategies and
  optimal strategy synthesis.

\section{Experiment equivalence} \label{sec:expeq}

We start with a formal definition of equivalence of two experiments.
The section continues with our suggestion on a method for equivalence testing
  based on isomorphism of labelled graphs.
This method is crucial for the algorithms presented in the following sections.

\begin{definition}[Experiment equivalence] \label{def:expeq}
Let $\exp\in\Exp$ be an experiment and $\perm\in\Perm_\Var$ a variable permutation.
A $\perm$-symmetrical experiment to $\exp$ is an experiment
  $\exp^\perm\in\Exp$
  such that
  $\{\form^\perm \in\outcome(\exp)\} = \{\form\in\outcome(\exp^\perm)\}$.
Clearly, no $\perm$-symmetrical experiment to $e$ may exists.

A \emph{symmetry group} $\symg$ of a given game is
  the maximal subset of $\Perm_\Var$ such that for
  every $\perm\in\symg$ and $\exp\in\Exp$,
  there exists a $\perm$-symmetrical experiment to $\exp$.

Finally, an experiment $\exp_1\in\Exp$ is equivalent to $\exp_2\in\Exp$ with respect to $\form$,
  written $\exp_1\expeq{\form}\exp_2$,
  if and only if there exists a permutation $\perm\in\symg$ such that
 \[ \{ \form\wedge\formx \| \formx\in\outcome(\exp_1) \} \equiv
   \{ (\form\wedge\formx)^\perm \| \formx\in\outcome(\exp_2) \}. \]
\end{definition}

\begin{example}
Recall the running example from the previous chapter, introduced in \autoref{ex:run1}.
Experiment 23 is a $(x_1x_3)$-symmetrical experiment to 12, because for $\perm = (x_1x_3)$,
\begin{alignat*}{5}
\big\{ \;(&(x_1 \wedge \neg y) \vee (x_2\wedge y))^\perm,\;
   (&&(x_1 \wedge y) \vee (x_2\wedge\neg y))^\perm,\;
   (&&\neg (x_1  \vee x_2))^\perm \;\big\} = \\
\big\{ \;&(x_3 \wedge \neg y) \vee (x_2\wedge y),
   &&(x_3 \wedge y) \vee (x_2\wedge\neg y),
   &&\neg (x_3 \vee x_2)\; \big\}.
\end{alignat*}

In fact, for every experiment $e = (t, p)$ and every permutation $\perm$ stabilizing $y$,
  we can permute the parameters of $t$ accordingly and get a $\perm$-symmetrical experiment to $e$.
Therefore, the symmetry group $\symg$ is $\{\perm\in\Perm_\Var \| \perm(y) = y\}$.

Since $\symg$ is also the symmetry group of $\init$,
  all experiments of the same type are equivalent,
  and the quotient set of $E$ by $\expeq{\init}$
  has only two equivalence classes.
For a more complex example, let $\form = \init\wedge\neg(x_1\vee x_2)$.
Experiment $3124$ is now equivalent to $43$, with $\perm = \idperm_\Var$.
The corresponding formulas are equivalent even though they
  are syntactically different. \eqed
\end{example}

In the rest of the section, we suggest a method for testing
  whether two given experiments are equivalent with respect to a given formula.

First, we show a construction of the \emph{base graph} for a given game,
  automorphisms of which are a subset of the symmetry group $\symg$.
Then we describe the construction of the \emph{experiment graph} for a given experiment,
  which is build on top of the base graph.
  We prove
  that if the experiment graphs are isomorphic,
  the corresponding experiments are equivalent.

Recall that a \emph{labelled graph} is a triple $(V, E, l)$, where
  $(V, E)$ is a graph and $l: V->L$
  is a labelling function ($L$ being a set of labels).
Isomorphism of two labelled graphs is a bijection between their sets of vertices
that preserves edges and labels.

\subsection{Base graph construction}

The base graph for a game $\game = (\Var, \init, \Sigma, F, \Expt)$ is a labelled graph $B = (V,E,l)$ described below.
\begin{itemize}
\item There is a vertex for every proposition variable and every mapping, i.e. $V = \Var \cupdot F$.
\item A mapping is connected by edges with all variables in its value range, i.e. $(f, x) \in E$ if there is a symbol $a\in\Sigma$ such that $f(a) = x$,
\item Two variables are connected by an edge if they are values
  of different mappings on the same symbol of the alphabet,
  and these mappings appear in outcome formulas of the same parametrized experiment.
  Formally, $(x_1, x_2) \in E$ if there is a symbol $a\in\Sigma$ and mappings
  $f_1,f_2\in F$ such that $f_1(a) = x_1$, $f_2(a) = x_2$, and
  there is a parametrized experiment $t\in T$ and a number $k <= n_t$ such that
  both $f_1(\$k)$ and $f_2(\$k)$ appear in the outcome formulas of the
  parametrized experiment $t$.
\item
  The vertices corresponding to mappings have their own labels.
  The vertices corresponding to variables are labelled ``variable'',
  expect for the variables that appear directly in some outcome formula
  of a parametrized experiment. These have their own labels as well.
\end{itemize}

\begin{example} \label{ex:cc-runbase}
The base graph for the counterfeit coin problem with 4 coins is shown in
\autoref{fig:base-graph} on the left.
Note that vertices $y$ and $f_x$ have separate labels while
  other vertices are labelled ``variable''.

\begin{figure}[ht]
\begin{center}
\includegraphics[width=.6\textwidth]{pictures/base-graph.pdf}
\caption{Base graph for the counterfeit coin problem with 4 coins (left) and\\
  for Mastermind with 3 pegs and 3 colours (right).}
\label{fig:base-graph}
\end{center}
\end{figure}
\end{example}

A more complicated example is the base graph for Mastermind with 3 pegs and 3 colours,
  shown on the right-hand side.
The vertices $f_1, f_2, f_3$ have separate labels, all other vertices are labelled ``variable''.
For simplicity, we leave out the symbol $x$ in the figure, e.g.
  write $1A$ instead of $x_{1A}$.\eqed

\begin{lemma}\label{lma:autobase}
Let  $\perm$ be an automorphism of $B$. Then $\perm|_\Var \in\symg$.
\end{lemma}

\begin{proof}
Let $\perm$ be an automorphism of $B$ and $(t, p)$ an experiment with a parametrization $p=p_1p_2...p_n$.
We show that there exists a $\perm$-symmetrical experiment to $(t, p)$.

Let $F_i\subseteq F$ be a set of mappings that are present in some outcome formula
  of $t$ with parameter $\$i$.
The vertices $f(p_i)$ for $f\in F_i$ form a clique in $B$ and so must the vertices
$\perm(f(p_i))$ for $f\in F_i$.
Since mappings $F$ have pairwise disjoint images, two variables $x_1, x_2$
  can be connected by an edge only if there is a symbol $k\in\Sigma$ and
  mappings $f,g\in F$ such that $f(k)=x_1$, $g(k)=x_2$.

We define $r_i$ as a symbol of $\Sigma$ that satisfies $f(r_i) = \perm(f(p_i))$ for some $f\in F_i$.
There always exists such $r_i$, because $f(p_i)$ cannot be mapped to a vertex that is not connected to $f$.
Due to the property above,
  if $f(r_i) = \perm(f(p_i))$ holds
  for \emph{some} $f\in F_i$,
  it holds for \emph{all} $f\in F_i$ and the definition is thus correct.

Now, consider the experiment $(t, r)$, where $r=r_1r_2,...r_n$.
All variables appearing directly in the parametrized formula are stabilized by $\perm$ and
  for all expressions $f(\$i)$ it holds $f(r_i)=\perm(f(p_i))$ by the construction of $r_i$,
  which means that $(t, r)$ is $\perm$-symmetrical to $(t, p)$. \qed
\end{proof}

\subsection{Experiment graph}

Let $\form\in\Form_X$ be a formula.
An \emph{$x$-rooted tree of $\form$}
  is a graph created from the syntax tree of $\form$
  by unification of the leaves that correspond to the same variables
  and adding a special vertex with label $x$ that is connected to the root
  of the syntax tree, i.e. to the top-level operator of $\form$.
Other vertices of the graph are labelled by their type (e.g. ``variable'', ``and-operator'', etc.)

In this construction, we need the trees of two formulas to be isomorphic if
  and only if the formulas are syntactically equivalent.
This clearly holds if all the operators are commutative.
As the only non-commutative operator is implication, we substitute
subformulas of the form $\form -> \formx$ with an equivalent formula $\formx \vee \neg\form$.

Let $B$ be the base graph for the given game, $\form\in\Formr$ some partial knowledge
  and $e$ an experiment.
The experiment graph $B_{\form, e}$ is constructed as follows.
\begin{itemize}
\item Begin with the graph $B$.
\item Add the ``knowledge''-rooted tree of $\form$.
\item For each outcome $\formx\in\outcome(e)$, add the ``outcome''-rooted tree of $\formx$.
\end{itemize}

\begin{theorem} \label{thm:isoequiv}
If $B_{\form, e_1}$ is isomorphic to $B_{\form, e_2}$, then
 $e_1 \expeq{\form} e_2$.
\end{theorem}

\begin{proof}
Let $\permx$ be the graph isomorphism of $B_{\form, e_1}$ and $B_{\form, e_2}$ and let
  $\perm = \permx|_\Var$, considered as a permutation of $\Var$.
Since $B$ is the vertex-induced subgraph of both $B_{\form,e_1}$ and $B_{\form, e_2}$ by
  the set of vertices $\Var\cupdot F$, $\perm$ is a member of $\symg$ by \autoref{lma:autobase}.

The isomorphism $\permx$ maps the only ``knowledge''-labelled vertex in the first graph
  to the only ``knowledge''-labelled vertex in the second graph,
  which implies the equivalence of the formulas, $\form^\perm \equiv \form$.
Similarly, ``outcome''-labelled vertices are mapped to ``outcome''-labelled vertices,
  which means that $\{\formx^\perm \| \formx\in\outcome(e_1)\} = \outcome(e_2)$.
This is sufficient for the experiments to be equivalent with respect to $\form$.
  \qed
\end{proof}

\begin{example}
Recall the running example of the counterfeit coin problem with four coins.
Base graph for the game was shown in \autoref{ex:cc-runbase}.
Let $\form=\init\wedge\neg(x_1\vee x_2)$ be the accumulated knowledge
  of the solving process $(12, =)$ and let $e$ be the experiment $3124$.
The experiment graph $B_{\form, e}$ is shown in \autoref{fig:exp-graph};
  Ex$_1$ denotes the $\exactlyk{1}$ operator.

Unfortunately, the graph for experiment 43 is clearly not isomorphic to this graph,
although the experiments are equivalent with respect to $\form$.
We address this problem in the following.\eqed

\begin{figure}[tt]
\begin{center}
\includegraphics[width=.7\textwidth]{pictures/exp-graph.pdf}
\caption{Experiment graph for 3124 with knowledge $\init\wedge\neg(x_1\vee x_2)$.}
\label{fig:exp-graph}
\end{center}
\end{figure}
\begin{figure}[t]
\begin{center}
\includegraphics[width=.7\textwidth]{pictures/exp-graph-sim.pdf}
\caption{Simplified experiment graph for 3124 with knowledge $\init\wedge\neg(x_1\vee x_2)$.}
\label{fig:exp-graph-sim}
\end{center}
\end{figure}
\end{example}
\subsection{Improvement by fixed variables}

The previous example shows that the method explained above
  does not detect some basic equivalences.
To address the problem, we suggest the following improvement to the construction
  of $B_{\form, e}$.
\begin{enumerate} \itemsep 2pt
\item Compute fixed variables of the formula $\form$ using a SAT solver.
\item Simplify the formula $\form$ with the knowledge of its fixed variables.
\item Simplify the outcomes of $e$, formulas $\formx\in\outcome(e)$, with
  the knowledge of fixed variables of $\form$.
\item Construct the graph as described above.
\item Label the vertices corresponding to the fixed variables with the label ``false'' or ``true'',
  according to their fixed value.
\end{enumerate}

As the simplified formulas are equivalent to the original formulas,
\autoref{thm:isoequiv} also holds if the graphs $B_{\form, e_1}$, $B_{\form, e_2}$
are constructed with this approach.

\begin{example}
Let us apply the suggested improvement on the previous example.
The formula $\form = \init\wedge\neg(x_1\vee x_2)$ fixed variables
$x_1$ and $x_2$ to false.
\autoref{fig:exp-graph-sim} shows the constructed experiment graph
  after the simplification of the formulas.

The vertices $x_1$ and $x_2$ are now labelled ``false'' and are connected
  only to the vertex $f_x$.
Compare the structure with the graph in \autoref{fig:exp-graph}.
Note that the graph is now isomorphic to the graph of the experiment $43$.\eqed
\end{example}

\autoref{alg:noneqexp} describes the elimination of equivalent experiments with respect to a formula $\form$,
  which is a straightforward application of the method described in this section.
We assume there is a tool available for construction of the canonical labelling
  of a given graph,
  which is used to decide graph isomorphism.

\begin{algorithm}[!ht]
\caption{Elimination of equivalent experiments}
\KwIn{formula $\form$}
\KwOut{set $S\subseteq E$, such that $\forall e\in E \;\exists s \in S.\;e\expeq{\form}s$}
\label{alg:noneqexp}
\DontPrintSemicolon
$B <-$ construct the base graph for the game\;
$fixed <-$ compute fixed variables of $\form$ using a SAT solver\;
$\form' <- $ substitude values for fixed variables in $\form$ and simplify\;
Label the vertices in $B$ corresponding to the fixed variables with their fixed value\;
Add the ``knowledge''-rooted tree of $\form'$ to $B$\;
$S <- \emptyset$\;
$hash <- $ an empty hash table for graphs\;
\For{$e\in E$}{
  $B_e <- $ clone $B$\;
  \For{$\formx\in\outcome(e)$} {
    $\formx' <- $ substitude values for $fixed$ in $\formx$ and simplify\;
    Add the ``outcome''-rooted tree of $\formx'$ to $B_e$
  }
  $B_e <- $ canonize $B_e$\;
  \If{$B_e$ is not present in $hash$}{
    $hash$.insert($B_e$)\;
    $S <- S \cup \{e\}$\;
  }
}
\Return{$S$}
\end{algorithm}

\pagebreak
\section{Well-formed check}

Experiment equivalence can be used during the verification
  that a given game is well-formed, as stated by the following lemma.

\begin{lemma} \label{lma:well-formed}
  Let $S\subseteq E$ be a subset of experiments
  such that for every $e\in E$, there exists $s\in S$
   such that $e\expeq{\init}s$.
  If the formula $\init ==> \exactlyk{1}(\outcome(e'))$ is a tautology for
  all $s\in S$, then the game is well-formed.
\end{lemma}

\begin{proof}
Assume by contradiction that the game is not well formed, i.e.
  there is $e\in E$ and $v \in \Vals$ such that the number of
  formulas in $\outcome(e)$ satisfied by $v$ is not equal to one.

If $e\in S$, the formula $\init ==> \exactlyk{1}(\outcome(e'))$ is not
  satisfied by $v$. Contradiction.

Otherwise, there exists $s\in S$ such that $e\expeq{\init}s$, which means that
  there exists $\perm\in\Perm_\Var$ such that
$\{ \init\wedge\formx \| \formx\in\outcome(e) \} =
 \{ (\init\wedge\formx)^\perm \| \formx\in\outcome(s) \}$.
Since $\init ==> \exactlyk{1}(\outcome(s))$ is a tautology,
  the permuted formula
  $\init^\perm ==> \exactlyk{1}(\formx^\perm \| \formx\in\outcome(s))$
  is a tautology as well.
Therefore, exactly one formula from the set
 $ \{ (\init\wedge\formx)^\perm \| \formx\in\outcome(s) \}$
 is satisfiable and the same holds for
  $\{ \init\wedge\formx \| \formx\in\outcome(e) \}$,
 which implies that
 $\init ==> \exactlyk{1}(\outcome(e))$
 is a tautology. \qed
\end{proof}

\section{Analysis of one-step look-ahead strategies}

The following lemma gives us a right to disregard equivalent experiments
  during the analysis of some one-step look-ahead strategies.

\begin{lemma}
Let $f: 2^\Formr ->\Rset$ be a function such that
  $f(\formset) = f(\{\form^\pi \| \form\in\formset\})$ for any
  $\formset\subseteq\Formr$ and $\pi\in\Perm_\Var$ and
  let $\eord$ be a total order of $E$.
Let $\stg$ be the one-step look-ahead strategy with respect to $f$ and $\eord$, and
let $\form$ be a formula.
Suppose there are experiments $e_1$, $e_2$ such that $e_1\expeq{\form}e_2$ and $e_1\eord e_2$.
Then $\stg(\form) \not= e_2$.
\end{lemma}

\begin{proof}
If follows directly from \autoref{def:expeq} and the property of $f$ that
the function $f$ have the same value on $e_1$ and $e_2$, i.e.
\[f(\{\form\wedge\formx\| \formx\in\outcome(e_1)\}) =
 f(\{\form\wedge\formx\| \formx\in\outcome(e_2)\}).\]
Since $e_1 \eord e_2$, the strategy always prefers $e_1$ to $e_2$. \qed
\end{proof}

\vspace{-5mm}
\begin{algorithm}[ht]
\caption{Analysis of a one-step look-ahead strategy}
\label{alg:stganalysis}
\DontPrintSemicolon
\KwIn{function $f: 2^\Formr -> \Rset$, total order $\eord$ of $E$}
\KwOut{($w,a$), where $w$ and $a$ is the worst-case and the average-case number of experiments performed by the strategy}
$globalsum <- 0$\;
$globalmax <- 0$\;
\textsc{Analyse}($\init$, 1)\;
\Return$ (globalmax,\; globalsum \;/\; \numval\init)$\;\medskip
\setcounter{AlgoLine}{0}
\SetKwProg{optfun}{Function}{}{}
\optfun{\textsc{Analyse}\textnormal{($\form$, $depth$)}}{
$choice <- $None\;
$bestvalue <- \infty$\;
$S <- $ eliminate equivalent experiments by running \autoref{alg:noneqexp} on $\form$,
  where the experiments are considered in the order given by $\eord$\;
\For{$e\in S$ (variant 1) or $e\in E$ (variant 2)}{
  $value <- f(e)$
  \If{$value < bestvalue$} {
    $choice <- e$\;
    $bestvalue <- value$\;
  }
}
\For{$\formx\in\outcome(e)$}{
  \lIf{not $\SAT{\form\wedge\formx}$}{continue}
  \eIf{$\numval(\form\wedge\formx) = 1$}{
    $globalsum <- globalsum + depth$\;
    $globalmax <- max(globalmax, depth)$\;
  }{
    \textsc{Analyse}($\form\wedge\formx$, $depth + 1$)\;
  }
}
}{}
\end{algorithm}

Note that all one-step look-ahead strategies discussed in \autoref{sec:oslas}
  satisfy the condition of the lemma.
In general, any function based on satisfiability, the number of models and/or
  the number of fixed variables of the formulas will satisfy this requirement as
  these function are permutation independent.

A recursive approach for the analysis of one-step look-ahead strategies
  is shown in \autoref{alg:stganalysis}.
There are two options in line $5$ of the \textsc{Analyse} function.
The first is to use the algorithm to eliminate equivalent experiments and thus
  evaluate the strategy only on a subset of experiments.
The second is to go through all possible experiments.

In general, it cannot be said which variant is faster.
This depends on the ratio between the time
  needed for graph canonization and the
  time needed for strategy evaluation.

\section{Optimal strategy synthesis}

We suggest a method for worst-case and average-case
  optimal strategy synthesis based on backtracking.
In every state, we consider all possible experiments
  and compute the number of steps we need if we start with
  this experiment.
Our goal in this section is to prove that
  it is enough to analyse only one experiment from each equivalence class,
  as equivalent experiments give the same results.

\newcommand{\optval}{\kappa}
\newcommand{\optexp}{\varepsilon}
\newcommand{\optvale}{\kappa_\textrm{exp}}
\newcommand{\optexpe}{\varepsilon_\textrm{exp}}

First, let us define $\optval(\form)$ and $\optvale(\form)$ as
 the optimal number of experiments needed to reveal the secret code
  when starting with knowledge $\form$
  in the worst-case and in the average-case, respectively.
We can say that $\optval(\form)$ ($\optvale(\form)$) is
  the number of experiments of
  a worst-case (average-case) optimal strategy
  if we change the initial constraint of the game to $\form$.

Similarly, we define $\optval(\form, e)$ and $\optvale(\form, e)$ as
  the optimal number of experiment needed to reveal the secret code
  when starting with knowledge $\form$ and
  with $e$ as the first experiment.

There is an obvious relationship between $\optval(\form)$ and $\optval(\form,e)$
  and between $\optvale(\form)$ and $\optvale(\form,e)$.
For any $\form\in\Formr$,
\begin{equation}
\optval(\form) = \min_{e\in\Exp}\optval(\form, e),\textrm{ and }
\hspace{1cm}
\optvale(\form) = \min_{e\in\Exp}\optvale(\form, e).
\label{opttriv}
\end{equation}

Further, we can compute $\optval(\form, e)$ and $\optvale(\form, e)$
  from the optimal values for the subproblems after the first experiment.
These relationships are based on the definitions of the worst-case
  and average-case number of experiments of a strategy
  ($\lenmax{\stg}$ and $\lenexp{\stg}$).
For any $\form\in\Formr$ and $e\in E$,
\begin{align}
\optval(\form, e) &= \left\{\begin{array}{ll}
 0 & \textrm{ if }\numval{\form} = 1, \\
 \infty & \textrm{ if }\exists\formx\in\outcome(e).\; \form\wedge\formx\equiv\form, \\
 1 + \max_{\formx\in\outcome(e)}\optval(\form\wedge\formx) &
 \textrm{ otherwise. }
\end{array}\right.\label{optval}\\
\optvale(\form, e) &= \left\{\begin{array}{ll}
 0 & \textrm{ if }\numval{\form} = 1, \\
 \infty & \textrm{ if }\exists\formx\in\outcome(e).\; \form\wedge\formx\equiv\form, \\
 1 + \frac{\sum_{\formx\in\outcome(e)}\numval{(\form\wedge\formx)}\cdot\optvale(\form\wedge\formx)}{\numval{\form}} &
 \textrm{ otherwise. }
\end{array}\right.\label{optvale}\\
\end{align}

Let us now define the sets of optimal choices in a state.
For a $\form\in\Formr$, we define
\begin{align*}
\optexp(\form) &= \{ e\in E \| \forall e'\in E.\; \optval(\form, e) <= \optval(\form, e') \},\textrm{ and }\\
\optexpe(\form) &= \{ e\in E \| \forall e'\in E.\; \optvale(\form, e) <= \optvale(\form, e') \}.
\end{align*}

The following lemma is a straightforward consequence of the definitions of $\optval$ and~$\optexp$.

\begin{lemma}
If $\stg$ is a strategy such that $\stg(\form)\in\optexp(\form)$ for every $\form\in\Formr$,
 $\stg$ is worst-case optimal.
Similarly, if $\stg'$ is a strategy such that $\stg'(\form)\in\optexpe(\form)$ for every $\form\in\Formr$,
 $\stg'$ is average-case optimal.
\end{lemma}

Now, we are ready for the main theorem of this section.
The first part gives us a right to compute the value of $\optval$
  on symmetrical formulas only once.
The second part
  allows us to consider
  only one experiment from
  each equivalence class of $E/\expeq{\form}$
  in every state.
The exact algorithm for optimal strategy synthesis
  with further optimizations is described in \autoref{s:cobra-modes}.

\begin{theorem}
For every $\form\in\Formr$,
\begin{enumerate}
\item $\optval(\form) = \optval(\form^\perm)$ and $\optvale(\form) = \optvale(\form^\perm)$ for all $\perm\in\symg$, and
\item if $e_1\expeq{\form} e_2$, then $e_1\in\optexp(\form) \Leftrightarrow e_2\in\optexp(\form)$ and
  $e_1\in\optexpe(\form) \Leftrightarrow e_2\in\optexpe(\form)$.
\end{enumerate}
\end{theorem}

\begin{proof}
The proof for the worst case ($\optval, \optexp$)
  and for the average case ($\optvale, \optexpe$) is exactly the same,
  so we show only the proof for the worst case.

Since $\perm\in\symg$, there exists a $\perm$-symmetrical experiment $e^\perm$
  to $e$ for every $e\in E$.
Recall that $\outcome(e^\perm) = \{ \formx^\perm \| \formx\in\outcome(e)\}$.
We show by induction on the number of models of $\form$
  that $\optval(\form, \exp) = \optval(\form^\perm, \exp^\perm)$,
  which is sufficient for the first part.

As $\numval{\form} = \numval{\form^\perm}$, the statement follows directly from
  (\refeq{opttriv}) and (\refeq{optval}) for formulas with one model.
For the induction step, observe that
  $\numval(\form^\perm \wedge \formx^\perm) = \numval(\form\wedge\formx)$
  and, by the induction hypothesis,
  $\optval(\form^\perm \wedge \formx^\perm) = \optval(\form\wedge\formx)$
  if $\form \not\equiv \form\wedge\formx$.
The statement now follows from (\refeq{optval})
  as the right sides are equal.

For the second part, it suffices to prove that
  $\optval(\form, e_1) = \optval(\form, e_2)$.
As the experiments are equivalent, there exists a permutation $\perm\in\symg$,
 such that
 $\{\form\wedge\formx \|\formx\in\outcome(e_1)\} =
 \{(\form\wedge\formx)^\perm \|\formx\in\outcome(e_2)\}$.
The equation now follows from (\refeq{optval}) and the facts that
 $\numval\form=\numval\form^\perm$ and $\optval(\form) = \optval(\form^\perm)$
 (proven in the first part). \qed
\end{proof}

A recursive algorithm for computation of the value
  of the worst-case and the average-case optimal strategy,
  $\optval(\form)$ and $\optvale(\form)$ is shown in \autoref{alg:acopt}.
The lines marked with [W] applies only to the worst case,
  the lines marked with [A] applies only to the average case.

The algorithm makes use of the first part of the theorem by caching the results and
  checking that the function has not yet been called on the same or a symmetrical formula
  in the begging.
This is done similarly to the symmetry detection described in \autoref{sec:expeq}.
We construct the base graph of the game, add the ``knowledge''-rooted tree of $\form$,
  canonize the graph and compare with the graphs we have already seen.

Apart from the formula $\form$, the recursive function takes another argument, $opt$,
  which is used for branch pruning in the computation of the worst-case optimal strategy.
The value of $opt$ is an upper bound on $\optval(\form)$.
Therefore, if we are sure that $\optval(\form, e) > opt$ for a given experiment $e$,
  we can continue with the analysis of another experiment.
A lower bound on $\optval(\form, e)$ can be computed using \autoref{lma:lbound}.
The initial value of $opt$ should be $\infty$ or any known upper bound on $\optval(\init)$.

\begin{algorithm}[ht]
\caption{Computation of the worst-case (W) and the average-case~(A) optimal number of experiments.}
\label{alg:acopt}
\DontPrintSemicolon
\SetKwProg{optfun}{Function}{}{}
\optfun{\textsc{Optimum}\textnormal{($\form$, $opt$)}}{
\lIf{$\numval{\form} = 1$}{\Return{0}}
Compute a canonical form of $\form$. If the function has already been called
  on $\form$ or a symmetrical formula, use the cached result. \;
[W] $ lb <- \textsc{LowerBound}(\form)$\;
[W] \lIf{$lb > opt$}{\Return{$\infty$}}
$S <- $ compute a subset of experiments such that
  $e\in E ==> \exists e'\in S.\;e\expeq{\form}e'$\;
% $ b <- $ maximal number of satisfiable outcomes of an experiment in $S$\;
% $ lb <- \textsc{LowerBound}(\form, b)$\;
% \lIf{$lb > opt$}{\Return{$\infty$}}
\For{$s\in S$, ordered by $\max_{\formx\in\outcome(s)}\numval(\form\wedge\formx)$}
{
  \lIf{only one of $\form\wedge\formx$, $\formx\in\outcome(s)$ is satisfiable}{continue}
  $ val <- 0 $\;
  \For{$\formx\in\outcome(s)$} {
    \If{$\SAT{\form\wedge\formx}$}{
[W] $val <- \max( val, 1 + \textsc{Optimum}(\form\wedge\formx,\; opt-1))$\;
[A] $val <- val + \numval(\form\wedge\formx)\cdot
      (1 + \textsc{Optimum}(\form\wedge\formx,\; opt-1))$\;
    }
  }
  [A] $val <- val \;/\; \numval(\form)$\;
  \lIf{$val < opt$}{$opt <- val$}
}
Store the information that the value for $\form$ is $opt$\;
\Return{$opt$}
}{}
\end{algorithm}

Note that the order of the experiments in line 7
  is not necessary for the correctness of the algorithm.
The idea here is to try to find a good experiment as soon as possible,
  so that we can prune some branches on the lower bound check.

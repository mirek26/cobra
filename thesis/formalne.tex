\chapter{Formal model}

In this chapter, we formally define code-breaking games
  within the framework of propositional logic,
  where we represent a secret code as a valuation
  of propositional variables.
We discuss various strategies for code-breaking games and
  define a concept of experiment equivalence,
  which is fundamental for strategy synthesis and analysis.

\section{Notation and Terminology}
Let $\Form_\Var$ denote the set of propositional formulas over
  the set of variables $\Var$ and let
  $\Val_\Var$ be the set of valuations (boolean interpretations)
  of variables $\Var$.
Apart from standard logical operators, we allow $n$-ary numerical operators
  $\exactlyk{k}$, $\atleastk{k}$, $\atmostk{k}$.
For a valuation $v\in\Val_\Var$ the operator $\exactlyk{k}$ has the semantics
  $\val(\exactlyk{k}(\form_1,...,\form_n)) = 1$ if and only if
  $|\{i \| v(\form_i) = 1 \}| = k$.
The semantics of $\atmost$ and $\atleast$ is defined analogically.

Formulas $\form_0, \form_1 \in \Form_\Var$ are equivalent,
  written $\form_0 \equiv \form_1$, if
  $\val(\form_0) = \val(\form_1)$ for all $\val\in\Val_\Var$.
We say that \emph{$\val$ is a model of $\form$}
  or that \emph{$\val$ satisfies $\form$}
  if $\val(\form) = 1$.
For a formula $\form\in\Form_\Var$, let
  $\numval_{\!_\Var}{\form} = |\{ \val\in\Val_\Var \| \val(\form) = 1 \}|$
  be the number of models of $\form$.
We often omit the index $\Var$ if it is clear from the context.

% For any unary predicate $P$, $\#i\in A.P(i) = |\{ i\in A \| P(i)\}|$.
%   We often omit the ``$\in A$'' part and write only $\#i.P(i)$
%   if the range of $i$ is clear from the context.

The set of all permutations of a set $\Var$ (bijections $\Var->\Var$)
  is denoted by $\Perm_\Var$ and
  $\idperm_\Var$ is the identity permutation.

A \emph{partition} $P$ of a set $X$ is a set of disjoint subsets of $X$,
  union of which is equal to $X$.
Members of $P$ are called \emph{cells}.
Let $P(x)$ be the cell containing $x$,
  i.e. $P(x) = A$, where $A\in P$ and $x\in A$.

% něco na *, hranaté závorky

%-------------------------------------------------------------------------------
% DEF: CODE BREAKING GAME
\section{Formal definition}

A code-breaking game can be represented by a \emph{set of variables},
  \emph{initial constraint} (a formula that is guaranteed to be satisfied),
  and a set of \emph{allowed experiments}.
An experiment is defined by the set of possible outcomes in which it can result.
The outcomes are specified in the form of a propositional formula that
  represents the partial information
  that the codebreaker gains if the experiment results in the outcome.

The number of experiments in a code-breaking game is typically very large.
For example, in the counterfeit coin problem defined in \autoref{s:coins},
  experiments correspond to combinations of coins you put on the pans of the
  balance scale.
It can be calculated that there are 36,894 combinations for 12 coins.
However, most of them have the same structure,
  so it would be inefficient to specify them one by one.
Therefore we choose a compact representation with \emph{parametrized experiments},
  where parametrization is a fixed-length string over a defined alphabet.
This whole idea is formalized below.

\begin{definition}[Code-breaking game] \label{def:game}
A \emph{code-breaking game} is a quintuple
  $\game = (\Var, \init, \Sigma, F, \Expt)$, where
  \begin{itemize}
  \item $\Var$ is a finite set of propositional variables,
  \item $\init \in \Form_\Var$ is a satisfiable propositional formula,
  %\item $\Expt$ is a finite set of types of experiments,
  \item $\Sigma$ is a finite alphabet,
  \item $F$ is a collection of mappings of type $\Sigma -> \Var$,
  \item $\Expt$ is a set of \emph{parametrized experiments}, defined below.
  \end{itemize}
\end{definition}

\begin{definition}[Parametrized experiment] \label{def:game}
A \emph{parametrized experiment} for a game
  $\game = (\Var, \init, \Sigma, F, \Expt)$
  is a triple
  $\expt = (n, P, \outcome)$, where
  \begin{itemize}
  \item $n$ is the number of parameters of the experiment,
  \item $P$ is a partition of the set $\{1, ..., n\}$,
  \item $\outcome$ is a set of parametrized formulas, defined below.
  \end{itemize}
If $k$ and $l$ are in the same cell of the partition $P$, then the $k$-th and
the $l$-th parameter must be different.
We denote the components of a parametrized experiment $t\in\Expt$
  by $n_t$, $P_t$, and $\outcome_t$.
\end{definition}

\begin{definition}[Parametrized formula] \label{def-pform}
The set of all \emph{parametrized formulas} for a parametrized experiment
$t$ of a game $\game = (\Var, \init, \Sigma, F, \Expt)$
  is the set of
  all strings $\pform$ generated by the following grammar:
  $$ \pform ::= x \| f(\$k) \| \pform \circ \pform \| O(\pform, ...)  \| \neg \pform,$$
  where $\pform, ...$ is a comma-separated list of $\pform$, $x\in\Var$, $f\in F$, $1<= k <= n_t$,
  $\circ\in\{\wedge, \vee, ==>\}$, and $O\in\{ \exactlyk{k}, \atmostk{k}, \atleastk{k} \| k\in\Nset\}$.
The special notation $\$k$ in $f(\$k)$ is used to denote the $k$-th parameter.
\end{definition}

The set $\Exp$ of all experiments in the game $\game$ is given by
\[
  \Exp =
     \big\{ (\expt, \param) \| t\in\Expt,\; \param \in\Sigma^{n_t},\;
     \forall x,y<={n_t}:\; P_t(x)=P_t(y) ==> \param[x] \not= \param[y] \big\}
\]

An experiment $e\in\Exp$ is thus a pair $(t, p)$, where
  $t$ is referred to as the \emph{type of the experiment},
 and $p$ is referred to as its \emph{parametrization}.

Let $e = (t, p)\in\Exp$ be an experiment,
  and $\pform\in\outcome_t$ a parametrized formula.
By $\pform(\param)$ we denote the application of
  the parametrization $\param$ on $\pform$,
  which is defined recursively on the structure of $\pform$
  in the following way:
\begin{align}
(x)(\param) &= x, \\
(f(\$k))(\param) &= f(\param[k]),\\
(\pform_1\circ\pform_2)(\param) &= \pform_1(\param) \circ \pform_2(\param),\\
O(\pform_1, ..., \pform_m)(\param) &= O(\pform_1(\param), ..., \pform_m(\param)),\\
(\neg\pform)(\param) &= \neg(\pform(\param)).
\end{align}

For the sake of simplicity, let us denote the set of possible outcomes for
  an experiment $\exp = (\expt, \param) \in \Exp$ by
  $\outcome(\exp) = \{ \pform(\param) \| \pform\in\outcome_\expt\}$.

\begin{example} \label{ex:run1}
Consider the counterfeit coin problem for 4 coins.

The counterfeit coin and its relative weight to others can be encoded
  as a valuation of variables $x_1, x_2, x_3, x_4$ and $y$,
  $v(x_i)$ being 1 if and only if the $i$-th coin is counterfeit and
  $y$ determining its relative weight
  ($v(y) = 0$ meaning lighter, $v(y) = 1$ meaning heavier).
The initial constraint $\init$ should capture the restriction that exactly one
  coin if counterfeit.
Therefore, let $\init$ be $\exactlyk{1}(x_1, x_2, x_3, x_4)$.

The experiments are parametrized by the coins on the pans of the balance scale.
Let $\Sigma = \{1, 2, 3, 4\}$ and $F = \{ f_x \}$ where $f_x$
maps $i$ to the corresponding variable $x_i$.

The first parametrized experiment $t$ is weighing one coin against one.
We need two parameters ($n_t = 2$),
  the first determining the coin on the left pan,
  the second determining the coin on the right pan that must be different
  from the first.
$P_t$ is therefore the trivial partition $\{\{1, 2\}\}$.

If the left pan is lighter, it is either the case that the
  coin on the left is underweight ($f_x(\$1) \wedge \neg y$)
  or the coin on the right is overweight ($f_x(\$2) \wedge y$).
If the right pan is lighter, we get symmetrical knowledge
  $(f_x(\$1)\wedge y) \vee (f_x(\$2)\wedge\neg y)$.
If both sides weigh the same, the counterfeit coins is not present on either pan
  and we can conclude $\neg f_x(\$1) \wedge \neg f_x(\$2)$.
To sum it up,
\begin{align*}
  t = \big(2,\; \big\{\{1,2\}\big\},\; \big\{ &
    (f_x(\$1)\wedge \neg y) \vee (f_x(\$2)\wedge y), \\
    & (f_x(\$1)\wedge y) \vee (f_x(\$2)\wedge\neg y), \\
    & \neg f_x(\$1) \wedge \neg f_x(\$2) \;\big\}\big).
\end{align*}

The second parametrized experiment is weighing two coins against two.
There are $4$ parameters, they must be pairwise distinct and the outcome
  formulas can be constructed analogically. \eqed
\end{example}


Note that the compact representation with parametrized experiments
  does not restrict the class of games that can fit \autoref{def:game}.
There can always be a parametrized experiment with no parameters for
  each actual experiment.

\begin{definition}[Solving process]
An \emph{evaluated experiment} is a pair $(e, \form)$,
  where $e\in\Exp$ and $\form\in\outcome(e)$.
Let us denote the set of evaluated experiments by $\Omega$.

A \emph{solving process} is a finite or infinite sequence
  of evaluated experiments.
\end{definition}

Let $\proc$ be a solving process.
For simplicity, we omit parentheses around the pairs and write
  $\proc = \exp_1, \form_1, \exp_2, \form_2, ...$.
Let
\begin{itemize}
\item $|\proc|$ denote the length of the sequence,
\item $\proc(k) = \exp_k$ denote the $k$-th experiment,
\item $\proc[k] = \form_k$ denote the $k$-th outcome,
\item $\proc[1..k] = \exp_1, \form_1, ..., \exp_k, \form_k$ denote the prefix of length $k$, and
\item $\aknow{\proc}{k} = \init \wedge \form_1 \wedge ... \wedge \form_k$
  denote the accumulated knowledge after the first $k$ experiments
  (including the initial constraint $\init$). For finite $\proc$,
  let $\tknow{\proc} = \aknow{\proc}{|\proc|}$ be the overall accumulated knowledge.
\end{itemize}

We denote the set of valuations that satisfy $\init$ by $\Vals = \{ \val\in\Val_\Var \| \val(\init) = 1 \}$
  and the set of \emph{reachable formulas} by $\Formr = \{ \tknow{\proc} \| \proc\in\Omega^* \}$.

\subsection{Course of the game}

Let us now describe the course of the game in the defined terms.
\begin{enumerate} \itemsep -3pt
\item The codemaker chooses a valuation $\val$ from $\Vals$.
\item The codebreaker chooses an experiment $e$ from $\Exp$.
\item The codemaker gives the codebreaker a formula
  $\form\in\outcome(e)$ that is satisfied by the valuation $\val$.
\item The evaluated experiment $(e, \form)$ is appended to the
  (initially empty) solving process $\proc$.
\item If $\numval{\tknow{\proc}} = 1$, the codebreaker can uniquely determine
the valuation $\val$ and the game ends. Otherwise, it continues with step 2.
\end{enumerate}
In order for the codemaker to always be able fulfil step 3,
  there must be a formula $\form\in\outcome(\exp)$ satisfied by any valuation.
Although it might make sense to allow multiple satisfied formulas, we restrict
  ourselves to games where the outcome is uniquely defined for a given valuation.

\begin{definition}[Well-formed game] \label{def:wellformed}
A code-breaking game is \emph{well-formed} if for all $\exp \in \Exp$,
\begin{equation}
\forall\val\in\Vals.\;
  \exists \textrm{ exactly one }
     \form\in\outcome(\exp)\;.\; \val(\form) = 1
\end{equation}
\end{definition}

In the sequel, we focus only on well-formed games and, by default,
  we assume a game is well-formed unless otherwise stated.

%-------------------------------------------------------------------------------
% EXAMPLE: FAKE-COIN PROBLEM
\subsection{Examples}
In the rest of this section, we show two ways of defining the counterfeit coin
  problem and a formal definition of Mastermind.
We do not provide formal definitions of other code-breaking games
  presented in \autoref{ch:games},
  however, a computer language for game specification
  that is based on this formalism is introduced in \autoref{ch:cobra},
  and specifications of all the code-breaking games
  in this language can be found in \autoref{app:games}.

\begin{example}[The counterfeit coin problem] \label{ex:cc1}
A formal definition of the counterfeit coin problem with 4 coins
  has already been introduced in \autoref{ex:run1}.
This is a straightforward generalization for $n$ coins.
We define a game $\mathcal{F}_n = (\Var, \init, \Sigma, F, \Expt)$ with
the following components.

\begin{itemize}
\item
$\Var = \{x_1, x_2, ..., x_n, y\}$.
  Variable $x_i$ tells whether coin $i$ is counterfeit,
  variable $y$ tells whether it is lighter or heavier.
\item
$\init = \exactlyk{1}(x_1, ..., x_n)$,
  saying that exactly one coin is counterfeit.

\item
$\Sigma = \{1, 2,...,n\}$, $F = \{ f_x \}$, where $f_x(i) = x_i$.
The experiments are parametrized with coins that are represented by numbers from 1 to $n$.

\item
$\Expt = \big\{ (2\cdot m,\; \{\{1,...,2m\}\},\; \outcome_m) \| 1 <= m <= n/2 \big\}$, where
\begin{flalign*}
\outcome_m = \big\{
& ((f_x(\$1) \vee ... \vee f_x(\$m)) \wedge \neg y) \vee ((f_x(\$m+1) \vee ... \vee f_x(\$2m)) \wedge y), \\
& ((f_x(\$1) \vee ... \vee f_x(\$m)) \wedge y) \vee ((f_x(\$m+1) \vee ... \vee f_x(\$2m)) \wedge \neg y), \\
& \neg (f_x(\$1) \vee ... \vee f_x(\$2m)) \big\}.
\end{flalign*}

For every $m\in\Nset$, $m <= n/2$, we have a parametrized experiment of weighing
  $m$ coins against $m$ coins.
It has $2m$ parameters, the first $m$ are put on the left pan, the last $m$ are put on the right pan.

There are 3 possible outcomes.
First, the left pan is lighter.
  This happens if the counterfeit coin is lighter and it appears
  among the first $m$ parameters,
  or if it is heavier and it appears among the last $m$ parameters.
Second, analogically, the right pan is lighter.
Third, both pans weigh the same if the
  counterfeit coin does not participate in the experiment.
\end{itemize}

For demonstration purposes, we show another possible formalization
  of the same problem.
Let $\mathcal{F'}_n = (\Var, \init, \Sigma, F, \expt)$ be a game
  with the following components.

\begin{itemize}
\item
$\Var = \{x_1, x_2, ..., x_n,\: y_1, y_2, ..., y_n\}$.
Variable $x_i$ tells that coin $i$ is lighter, variable $y_i$ tells that coin $i$ is heavier.
\item
$\init = \exactlyk{1}(x_1, ..., x_n, y_1, ..., y_n)$,
  saying that exactly one coin is odd-weight.
\item $\Sigma = \{1, 2,...,n\}$, $F = \{ f_x, f_y \}$, where $f_x(i) = x_i$, $f_x(i) = y_i$.
\item
$\Expt = \big\{ (2\cdot m,\; \{\{1,...,2m\}\},\; \outcome_m) \| 1 <= m <= n/2 \big\}$, where
\begin{flalign*}
\outcome(w_m) = \big\{ & f_x(\$1) \vee ... \vee f_x(\$m) \vee f_y(\$m+1) \vee ... \vee f_y(\$2m), & \\
& f_y(\$1) \vee ... \vee f_y(\$m) \vee f_x(\$m+1) \vee ... \vee f_x(\$2m), & \\
& \neg\; (f_x(\$1) \vee ... \vee f_x(\$2m) \vee f_y(\$1) \vee ... \vee f_y(\$2m)) \big\}. &
\end{flalign*}
In this formalization, the variables correspond one-to-one to possible codes,
 so the outcome formulas effectively list all possibilities. \eqed
\end{itemize}
\end{example}

%-------------------------------------------------------------------------------
% EXAMPLE: MASTERMIND 2

\begin{example}[Mastermind] \label{ex:form-mastermind}
Mastermind game with $n$ pegs and $m$ colours can be formalized as
a code-breaking game
$\mathcal{M}_{n,m} = (\Var, \init, \Sigma, F, \Expt)$
  with the following components.

\begin{itemize}
\item
$\Var = \{x_{i,j} \| 1<=i<=n, 1<=j<=m \}$.
  Variable $x_{i,j}$ tells whether there is colour $j$ at position $i$.
\item
$\init = \bigwedge\left\{
  \exactlyk{1} \{x_{i,j} \| 1<=j<=m\} \| 1<=i<=n\right\}$, saying that
  there is exactly one colour at each position.
\item $\Sigma = \{1,...,m\}$, \\
 $F = \{ f_1, ..., f_n \}$, where $f_i(c) = x_{i,c}$ for $1<=i<=n$, \\
 $\Expt = \{ (n, P, \outcome) \}$.\\
There is only one parametrized experiment with $n$ parameters corresponding
  to the colours.
All parameters can be the same,
  so the partition $P$ is the discrete partition $\{\{1\},...,\{n\}\}$.
\item $\outcome = \{ \texttt{Outcome}(b, w) \| 0<=b<=n, 0<=w<=n, b+w<=n \}$,
where $\texttt{Outcome}$ function is computed by the algorithm described below.
\end{itemize}

As described in \autoref{sec:mm},
  the outcome of an experiment corresponds
  to some maximal matching between
  the pegs in the code and the pegs in the guess.
The idea here is to generate a formula that asserts
  existence of such maximal matching
  with $b$ edges corresponding to black markers and
  $w$ edges corresponding to white markers.

The computation of \texttt{Outcome} $(b, w)$ is performed as follows.
First, we generate all admissible matchings.
Let $P = \{1,2,...,n\}$ be the set of positions.
\begin{itemize}
\item Select $B\subseteq P$ such that $|B| = b$.
  These are the positions at which the colour
  in the code matches the colour in the guess.
  They correspond to the black markers.
\item Select $W\subseteq P\times P$ such that $|W| = w$,
  $p_1(W)\cap B = \emptyset$, and $p_2(W)\cap B = \emptyset$,
  where $p_1$, $p_2$ are projections.
  These correspond to the white markers; $(i, j) \in W$ means that the colour
  at position $i$ in the guess is at position $j$ in the code.
\end{itemize}

Recall that $\inguess{i}$ denotes position $i$ in the guess
  and $\incode{i}$ denotes position $i$ in the code.
For a fixed combination $(B, W)$, we define matchin $M$ by
 $M = \{(\inguess{i}, \incode{i}) \| i\in B \} \cup
      \{(\inguess{i}, \incode{j}) \| (i, j)\in W \}$.
We construct a parametrized formula
  that asserts that $M$ is the maximal matching satisfying conditions in \autoref{sec:mm}
  for a guess $\$1,\$2,...,\$n$ and the code given by a valuation of the variables.
The formula has a form of a conjunction constructed in the following way.
\begin{itemize}
\item For $i\in B$, we add $f_i(\$i)$.
  This asserts that $(\inguess{i}, \incode{i})$ is an edge in the matching.
\item For $(i,j)\in W$, we add $f_j(\$i) \wedge \neg f_i(\$i) \wedge \neg f_j(\$j)$.
  This asserts that the colour $\$i$ is at position $j$ in the code and that
  $(\inguess{i}, \incode{i})$, $(\inguess{j}, \incode{j})$ cannot be edges
  in the matching.
\item For $(i,j)\in (P\setminus B\setminus p_1(W))
             \times (P\setminus B\setminus p_2(W))$, we add $\neg f_j(\$i)$.
  This asserts that no edge can be added and the matching is maximal.
\end{itemize}

The result of $\texttt{Outcome}(b, w)$ is a disjunction
  of all the conjunctions constructed in this way
  for all combinations of $B$ and $W$.
For example, for $n = 4$, $B = \{1\}$ and $W = \{2, 3\}$, the generated formula is
\[ f_1(\$1) \wedge f_3(\$2) \wedge \neg f_2(\$2) \wedge \neg f_3(\$3)
  \wedge \neg f_2(\$3) \wedge \neg f_2(\$4) \wedge \neg f_4(\$3) \wedge \neg f_4(\$4). \]

The number of combinations for $B$ and $W$ grows exponentially
  with $n$ and so does the size of generated formulas.
For $n = 4$, the result of $\texttt{Outcome}(1, 1)$
contains 24 clauses at the top level with 192 literals in total.\eqed

% For completeness, we show another way to formalize the Mastermind game,
%   which does not need algorithmic generation of the formulas.
% Let
%   $\mathcal{M'}_{n,m} = (\Var, \init, \Expt, \Sigma, \Exp, F, \outcome)$,
%   where

% \begin{itemize}
% \item $\Var$, $\init$, $\Sigma$, $F$ are defined as before.
% \item
% $\Expt = \{ g_{k_1,...,k_m} \| k_i \in \{1,...,n\}, \sum_ik_i = n \}$,\\
% $\Sigma = C$, \\
% $\Exp = \{(g_{k_1,...,k_m}, \param) \| \param\in\Sigma^{n},
%   \numpred{i}{\param[i]=j}=k_j\}$.\\
% The type $g_{k_1,...,k_m}$ covers all the guesses in which the number of $j$-coloured pegs is $k_j$.
% Therefore, two guesses for which we use the same pegs (pegs are just shuffled) are of the same type,
% but if we change a peg for one with different colour, it is other type of experiment.

% \item
% $F = \{ f_1, ..., f_n \}$, where $f_i(c) = x_{i,c}$ for $1<=i<=n$,
% \vspace{-2mm}
% \begin{flalign}
% \outcome(& g_{k_1,...,k_n}) =  \Big\{ &\\
%  & \exactlyk{b}\{ f_i(\$i) \| 1<=i<=n \} \;\wedge & \label{eq:mm-blacks}\tag{1}\\
%  & \exactlyk{t}\bigcup
%       \big\{
%            \{ \atleast{l}(x_{1,j},...,x_{n,j}) \| 1 <= l <= k_j \}
%            \| 1<=j<=m
%       \big\} & \label{eq:mm-whites}\tag{2}\\
%   &\hspace{2cm} \| 0<=b<=t, 0<=t<=n\Big\}.
% \end{flalign}
% \end{itemize}

% Part \eqref{eq:mm-blacks} of the formula captures the number of
%   the black markers.
% Part \eqref{eq:mm-whites} captures the total number of markers.
% Indeed, we get $k$ markers for colour $j$
%   if and only if $k < k_j$ and there are
%   at least $k$ pegs of colour $j$ in the code, i.e. all the formulas
%   $\atleast{i}(x_{1,j},...,x_{n,j})$ are satisfied for $i <= k$.
% Note that since the number of pegs of each colour is fixed by the type and we
%   do not care about the exact positions, this part of the formula
%   is not parametrized. \eqed
\end{example}

\section{Strategies in general}

This section introduces the concept of a strategy for experiment selection.
We define worst-case and average-case number of experiments of a strategy
 and optimal strategies. Further, we examine several strategy classes.

\begin{definition}[Strategy]\label{def:strategy}
A \emph{strategy} is a function $\stg: \Omega^* -> \Exp$,
  determining the next experiment for a given finite solving process.
\end{definition}

A strategy $\stg$ together with a valuation $\val\in\Vals$
  induce an infinite solving process
  \[
  \procstg{\stg}{\val} = \exp_1, \form_1, \exp_2, \form_2, ...,
  \]
  where
  $\exp_{i+1} = \stg(\exp_1, \form_1, ..., \exp_i, \form_i)$
  and
  $\form_{i+1}$ is the formula from $\outcome(\exp_{i+1})$
  satisfied by $\val$,
  for all $i\in\Nset$.
Note that thanks to the well-formed property,
  $\form_{i+1}$ is uniquely defined.

We define \emph{length} of a strategy $\stg$ on a valuation $\val$,
  denoted $\stglen{\stg}{\val}$,
  as the smallest $k\in\Nseto$ such that
  $\stgknow{\stg}{\val}{k}$ uniquely determines the code, i.e.
  \[
  \stglen{\stg}{\val} = \min \;\{ k\in\Nseto \| \numval{\stgknow{\stg}{\val}{k}} = 1 \}
  \]


The \emph{worst-case number of experiments} $\lenmax{\stg}$
  of a strategy $\stg$ is the maximal length of the strategy on a valuation $\val$,
  over all $\val\in\Vals$, i.e.
  \[
  \lenmax{\stg} = \max_{\val\in\Vals} \stglen{\stg}{\val}.
  \]

The \emph{average-case number of experiments} $\lenexp{\stg}$
  of a strategy $\stg$ is the expected number of experiments if the code
  is selected from models of $\init$ with uniform distribution, i.e.
  \[
  \lenexp{\stg} = \frac{
    \sum_{\val\in\Vals} \stglen{\stg}{\val}
    }{\numval{\init}}.
  \]

We say that a strategy $\stg$ \emph{solves the game} if $\lenmax{\stg}$ is finite.
Note that $\lenmax{\stg}$ is finite if and only if $\lenexp{\stg}$ is finite.
The game is \emph{solvable} if there exists a strategy that solves the game.

\medskip

\begin{definition}[Optimal strategy]
A strategy $\stg$ is \emph{worst-case optimal} if
  $\lenmax{\stg} <= \lenmax{\stg'}$ for any strategy $\stg'$.
A strategy $\stg$ is \emph{average-case optimal} if
  $\lenexp{\stg} <= \lenexp{\stg'}$ for any strategy $\stg'$.
\end{definition}

The following lemma provides us with a lower bound on the number of
experiments of a worst-case optimal strategy.

\begin{lemma} \label{lma:lbound}
Let $b = \max_{\expt\in\Expt} |\outcome(\expt)|$ be the maximal number of
  possible outcomes of an experiment. Then for every strategy $\stg$,
  \[
  \lenmax{\stg} >= \lceil \log_b(\numval{\init}) \rceil.
  \]
\end{lemma}

\begin{proof}
Let us fix a strategy $\stg$ and $k = \lenmax{\stg}$.
For an unknown model $\val$ of $\init$,
  $\stgknow{\stg}{\val}{k}$ can take up to
  $b^k$ different values.
By pigeon-hole principle, if $\numval{\init} > b^k$, there must be a valuation
  $v$ such that $\numval{\stgknow{\stg}{\val}{k}} > 1$.
This would be a contradiction with $k = \lenmax{\stg}$ and, therefore,
  $\numval{\init} <= b^k$, which is equivalent with the statement of the lemma.
  \qed
\end{proof}

\begin{lemma} \label{lma:accumulatedknowledge}
Let $\stg$ be a strategy and let $\val_1, \val_2 \in\Vals$.
If $\val_1$ is a model of $\stgknow{\stg}{\val_2}{k}$,
  then $\procstg{\stg}{\val_1}[1:k] = \procstg{\stg}{\val_2}[1:k]$.
\end{lemma}

\begin{proof}
Let $\proc_1 = \procstg{\stg}{\val_1}$, $\proc_2 = \procstg{\stg}{\val_2}$
and consider the first place where $\proc_1$ and $\proc_2$ differs.
It cannot be an experiment $\proc_1(i) \not= \proc_2(i)$ as they are both
  values of the same strategy on the same process:
$\proc_1(i) = \stg(\proc_1[1:i-1]) =
              \stg(\proc_2[1:i-1]) = \proc_2(i)$.

Suppose it is an outcome of the $i$-th experiment, $\proc_1[i] \not= \proc_2[i]$
  and $i <= k$.
Since $\val_1$ satisfies $\aknow{\proc_2}{k}$ and $i <= k$,
  it satisfies $\proc_2[i]$ as well.
However, $\val_1$ always satisfies $\proc_1[i]$ and
  both $\proc_1[i]$ and $\proc_2[i]$ are from the set
  $\outcome(\proc_1(i)) = \outcome(\proc_2(i))$.
Since there is exactly one satisfied experiment for each valuation in the set,
  $\proc_1[i]$ and $\proc_2[i]$ must be the same.
Contradiction. \qed
\end{proof}

\begin{example} \label{ex:run2}
Recall our running example of the counterfeit coin problem with 4 coins,
as defined in \ref{ex:run2}.

Consider a strategy $\stg$ defined as follows.
For simplicity, we denote experiments by their parametrizations only
  and the outcomes by a symbol $<$, $>$ and $=$,
  instead of the corresponding formula.
\[
\stg(\proc) = \left\{\begin{array}{ll}
13 & \textrm{ if } \proc = (12, <), \\
23 & \textrm{ if } \proc = (12, >),\\
14 & \textrm{ if } \proc = (12, =), (12, =), \\
34 & \textrm{ if } \proc = (12, =), (12, =), (14, =), \\
12 & \textrm{ otherwise.}
\end{array}\right.
\]

Let $\val\in\Vals$ be a valuation such that $v(x_3) = v(y) = 1$.
The induced solving process is
\[
\procstg{\stg}{\val} = (12, =), (12, =), (14, =), (34, >), (12, =), (12, =), ...
\]
The length of $\stg$ on $\val$ is 4, because $v$ is the only model of
  the accumulated knowledge after 4 experiments,
\[
\exactly_1(x_1,x_2,x_3,x_4) \wedge \neg(x_1 \vee x_2) \wedge \neg(x_1 \vee x_2)
\wedge \neg(x_1 \vee x_4) \wedge ((x_3 \wedge y) \vee (x_4 \wedge \neg y)).
\]

The strategy is intentionally inefficient and repeats the experiment $12$
if the outcome in the first step is `$=$'.
In fact, every valuation is discovered by $\stg$ in at most 4 experiments,
  so $\lenmax{\stg} = 4$.

\autoref{lma:lbound} gives us a lower bound $ \lceil\log_3(8)\rceil = 2$
 on the worst-case number of experiments of an optimal strategy.
However, we already know from \autoref{th:coins12} that the minimal number
  of experiments needed to reveal the code is 3.
\end{example}

\subsection{Non-adaptive strategies}

Non-adaptive strategies correspond to the well-studied problems of
  static Mastermind and
  non-adaptive strategies for
  the counterfeit coin problem\cite{mm-static}\cite{coins-nonadaptive}.
We define them here only to show the possibility of formulating the
  corresponding problems
  in our framework but we do not study them any further.

\begin{definition}[Non-adaptive strategy]
A strategy $\stg$ is \emph{non-adaptive} if it decides the next experiment
  based on the length of the solving process only, i.e.
  whenever $\proc_1$ and $\proc_2$ are processes such that
  $|\proc_1| = |\proc_2|$,
  then
  $\stg(\proc_1) = \stg(\proc_2)$.

Non-adaptive strategies can be considered functions $\stgx: \Nseto -> \Exp$,
where $\tau(|\proc|) = \stg(\proc)$.
\end{definition}

\subsection{Memory-less strategies}

The general definition of a strategy allows for the next experiment
  to depend on the exact history of the solving process,
  not only on the accumulated knowledge.
This is in a sense unintuitive, as
  the nature of code-breaking games is memory-less
  and the course of a game depends only on the accumulated knowledge.

\begin{definition}[Memory-less strategy]
A strategy $\stg$ is \emph{memory-less} if it decides the next experiment
  based on the accumulated knowledge only, i.e.
  whenever $\proc_1$ and $\proc_2$ are processes such that if
  $\tknow{\proc_1} \equiv \tknow{\proc_2}$
  then
  $\stg(\proc_1) = \stg(\proc_2)$.

Memory-less strategies can be considered functions
  $\stgx: \Formr -> \Exp$ such that
  $\form_1\equiv\form_2 ==> \stgx(\form_1)=\stgx(\form_2)$.
Then $\stg(\proc) = \stgx(\tknow{\proc})$.
\end{definition}

Note that the number of non-equivalent formulas over variable $\Var$
  is finite and, therefore, the number of memory-less strategies for a fixed
  code-breaking game
  is finite as well.

Now we prove some basic properties of memory-less strategies.
The following lemma says that once we do not get any new information
  from the experiment selected by a experiment,
  we never get any new information with the strategy.
Then, the theorem below proves that there exists an optimal
  memory-less strategy.

\begin{lemma}
Let $\stg$ be a memory-less strategy and $\val\in\Vals$.
If there exists $k\in\Nset$ such that
  $\numval{\stgknow{\stg}{\val}{k}} = \numval{\stgknow{\stg}{\val}{k+1}}$,
 then
  $\numval{\stgknow{\stg}{\val}{k}} = \numval{\stgknow{\stg}{\val}{k+l}}$
 for any $l\in\Nset$.
\end{lemma}

\begin{proof}
To simplify the notation, let $\know^k = \stgknow{\stg}{\val}{k}$.
There is a formula $\form\in\outcome(\know^k)$,
  such that $\know^{k+1} \equiv \know^{k} \wedge \form$.
Therefore, if $\know^{k+1}$ is satisfied by valuation $\val$, so must be $\know^{k}$.
Since $\numval{\know^{k}} = \numval{\know^{k+1}}$, the sets of
  valuations satisfying $\know^{k}$ and $\know^{k+1}$ are exactly the same
  and the formulas are thus equivalent.
This implies $\stg(\know^{k}) = \stg(\know^{k+1})$ and $\know^{k+2} \equiv \know^{k+1}\wedge \form \equiv \know^{k+1}$.

By induction,
  $\stg(\know^{k+l}) = \stg(\know^{k})$ and
  $\know^{k+l} \equiv \know^{k}$
  for any $l\in\Nset$.\qed
\end{proof}

\TODO{W-c vs A-c???}
\begin{theorem} \label{th:memless}
Let $\stg$ be a strategy.
Then there exists a memory-less strategy $\stgx$ such that
  $\stglen{\stg}{\val} >= \stglen{\stgx}{\val}$ for all $\val\in\Vals$.
\end{theorem}

\begin{proof}
Let us choose any total order $\form_1, \form_2, ...$ of $\Formr$ such that
  if $\form_i$ implies $\form_j$, then $i <= j$.
We build a sequence of strategies $\stg_0, \stg_1, \stg_2, ...$ inductively in the following way.
Let $\stg_0 = \stg$.
\begin{itemize}
\item If there is no $v\in\Vals, k\in\Nseto$ such that
  $\stgknow{\stg_{i-1}}{v}{k} \equiv \form_i$, select any $\exp\in\Exp$ and
  define $\stg_i$ by
\[
\stg_i(\proc) = \left\{
 \begin{array}{lll}
 \stg_{i-1}(\proc)  & \textrm{ if } \tknow{\proc}\not\equiv\form_i,\\
 \exp               & \textrm{ if } \tknow{\proc}\equiv\form_i.
 \end{array}
 \right.
\]
Clearly, all induced solving processes for $\stg_i$ and $\stg_{i-1}$ are the same
  and $\stglen{\stg_i}{v} = \stglen{\stg_{i-1}}{v}$.

\item If there exists $v\in\Vals, k\in\Nseto$ such that
  $\stgknow{\stg_{i-1}}{v}{k} \equiv \form_i$, choose the largest $l$ such that
  $\stgknow{\stg_{i-1}}{v}{l} \equiv \form_i$ and define
\[
\stg_i(\proc) = \left\{
 \begin{array}{lll}
 \stg_{i-1}(\proc)            & \textrm{ if } \tknow{\proc}\not\equiv\form_i,\\
 \procstg{\stg_{i-1}}{v}(l)   & \textrm{ if } \tknow{\proc}\equiv\form_i.
 \end{array}
 \right.
\]
First we prove that this definition is correct.
Let $v_1, v_2, k_1, k_2$ be such that
  $\stgknow{\stg_{i-1}}{v_1}{k_1}\equiv\form_i\equiv\stgknow{\stg_{i-1}}{v_2}{k_2}$.
Take $l_1, l_2$ as the largest numbers such that
  $\stgknow{\stg_{i-1}}{v_1}{l_1}\equiv\form_i\equiv\stgknow{\stg_{i-1}}{v_2}{l_2}$.
Since $v_1$ satisfies $\stgknow{\stg_{i-1}}{v_2}{l_2}\equiv\form_i$,
  then $\procstg{\stg_{i-1}}{v_2}[1:l_2] = \procstg{\stg_{i-1}}{v_1}[1:l_2]$
  by \autoref{lma:accumulatedknowledge}.
The same holds for $l_1$ which means that $l_1 = l_2$ and
  $\procstg{\stg_{i-1}}{v_1}(l_1) = \procstg{\stg_{i-1}}{v_1}(l_2)$, which
  proves that the definition of $\stg_i$ is independent of the exact choices
  of $v$ and $k$.

Now $\stglen{\stg_i}{v} = \stglen{\stg_{i-1}}{v} - (l-k)$, where
  $k$ and $l$ is the smallest and the largest number such that
  $\stgknow{\stg_{i-1}}{v}{k}\equiv\form_i $ and
  $\stgknow{\stg_{i-1}}{v}{l}\equiv\form_i $, respectively,
  because
  $\procstg{\stg_{i-1}}{v}(l) = \procstg{\stg_{i}}{v}(k)$ and due to the ordering,
  the rest of the process is independent of the beginning.
\end{itemize}

The last strategy of the sequence is clearly memory-less and satisfies the
  condition in the lemma. \qed
\end{proof}

% \begin{lemma}
% Let $b = \max_{\expt\in\Expt} |\outcome(\expt)|$ be the maximal number of
%   possible outcomes of an experiment.
% If for any $\form\in\Formr$,
% \[
%   \exists\exp . \max_{\formx\in\outcome(\exp)} \numval{(\form\wedge\formx)} =
%   \left\lceil \frac{\numval{\form}}{b} \right\rceil,
% \]
% then a greedy strategy $\stg$ is optimal and
% \[
%   \lenmax{\stg} = \lceil \log_b(\numval{\init}) \rceil.
% \]
% \end{lemma}

% \begin{proof}
% \TODO{Napsat důkaz.}
% \end{proof}

\begin{example}
Recall the game and the strategy $\stg$ from \autoref{ex:run2}.
The strategy is clearly not non-adaptive, as
  $\stg((12, <)) \not= \stg((12, >))$.
It is neither memory-less as
  $\stg((12, =)) \not= \stg((12,=),(12,=))$ but
  the accumulated knowledge of the solving processes is the same.

Consider a non-adaptive strategy
 $\stgx:\; 1 \mapsto 12,\; 2\mapsto 13,\; 3\mapsto 14$.
If the counterfeit coin is among the first three,
  it is discovered by the strategy in two experiments.
If the counterfeit coin is coin 4, it requires three experiments.
Hence $\lenmax{\stgx} = 3$ and the value of $\stgx$ on
  greater numbers is irrelevant.

If we apply the construction in \autoref{th:memless} on $\stg$,
we get a memory-less strategy $\stg'$, given by

\[
\stg'(\form) = \left\{\begin{array}{ll}
13 & \textrm{ if } \form \equiv (x_1 \wedge \neg y) \vee (x_2 \wedge y), \\
23 & \textrm{ if } \form \equiv (x_1 \wedge y) \vee (x_2 \wedge \neg y),\\
14 & \textrm{ if } \form \equiv \neg x_1 \wedge \neg x_2, \\
34 & \textrm{ if } \form \equiv \neg x_1 \wedge \neg x_2 \wedge \neg x_4, \\
12 & \textrm{ otherwise.}
\end{array}\right.
\]

Notice that the valuation $v$ with $v(x_3) = v(y) = 1$ is discovered in
  3 experiments as the strategy does not repeat the experiment 12 now.
Therefore, $\lenmax{\stg'} = 3$.

Both strategies $\tau$ and $\stg'$ are worst-case optimal. \eqed
\end{example}

\section{One-step look-ahead strategies} \label{sec:oslas}

Specification of a strategy in general can be very complicated.
In this section, we study a subclass of memory-less strategies that we call
  \emph{one-step look-ahead}.
These strategies select an experiment that
  minimizes the value of a given function
  on the set of possible knowledge in the next step.

\newcommand{\formset}{\Psi}
\begin{definition}[One-step look-ahead strategy]\label{def:oslas}
Let $f$ be a function of type $2^{\Formr} -> \Rset$.
A one-step look-ahead strategy with respect to $f$ is
  a memory-less strategy such that
  for every $\form\in\Form_X$ and $\exp'\in\Exp$,
\[
f(\{\:\form \wedge \formx \| \formx\in\outcome(e) \:\}) <=
  f(\{\: \form \wedge \formx \| \formx\in\outcome(e') \:\}).
\]
\end{definition}

Note that one-step look-ahead strategy with respect to $f$ is not unique.
For some formulas, there can be more experiments with the same value of $f$.
To uniquely specify a strategy, we must provide the function $f$ and
  a resolution method for these ambiguous states.
Typically, we specify a total order on experiments and select the least
  experiment in the order satisfying the condition of \autoref{def:oslas}.

A few one-step look-ahead strategies for Mastermind
  have been already introduced in \autoref{sec:mm}.
We now define them formally in the general code-breaking games.
In the Mastermind case,
  the experiments are ordered lexicographically by the colour combination.

\begin{description}
\item[Maximal number of models.]
This strategy minimizes the worst-case number of remaining codes.
For Mastermind, this was suggested by Knuth\cite{mm-knuth}.
\[
f(\formset) = \max_{\form\in\formset} \numval{\form}
\]

\item[Expected number of models.]
This strategy minimizes the expected number of remaining codes.
  For Mastermind, this was suggested by Irwing\cite{mm-expnum}.
\[
f(\formset) = \frac{\sum_{\form\in\formset}(\numval{\form})^2}{\sum_{\form\in\formset} \numval{\form}}
\]

\item[Entropy of the number of models.]
This strategy maximizes the entropy of the numbers of remaining codes,
For Mastermind, this was suggested by Neuwirth\cite{mm-entropy}.
\[
f(\formset) = \sum_{\form\in\formset} \frac{\numval{\form}}{N} \cdot \log \frac{\numval{\form}}{N}
  \textrm{, where } N = \sum_{\form\in\formset}\numval{\form}
\]

\item[Number of satisfiable outcomes.]
This strategy maximizes the number of possible outcomes.
For Mastermind, this was suggested by Kooi\cite{mm-mostparts}.
\[
f(\formset) = - \:|\{ \form \| \form\in\formset, \;\SAT{\form} \}|
\]
\end{description}

\newcommand{\fixed}{\#_\textrm{fixed}\:}
We suggest and analyse one-step look ahead strategies based on fixed variables.
Let
\[
\fixed\form = |\{ x\in\Var \| \forall v.v(\form)=1 ==> v(x)=1 \}
                  \cup\{ x\in\Var \| \forall v.v(\form)=1 ==> v(x)=0 \}|
\]
be the number of variables that have same value in all models of $\form$.
Note that while the aforementioned strategies does not depend on the exact
  formalization of a problem, the number of fixed variables may differ for
  different encodings.
For example, the choice of the following strategies in \autoref{ex:cc1} differs
  for the two possible formalisations.

\begin{description}
\item[Minimal number of fixed variables.]
\[
f(\formset) = -\min_{\form\in\formset} \fixed{\form}
\]
\item[Expected number of fixed variables.]
\[
f(\formset) = -\frac{\sum_{\form\in\formset}\numval{\form}\cdot\fixed{\form}}{\sum_{\form\in\formset}\numval{\form}}
\]
\end{description}

\begin{example}
Recall \autoref{ex:run1} and consider two experiments in the first step.

First, consider an experiment of weighing coin 1 against coin 2.
All the 3 outcomes are satisfiable, the number of models is
  2, 2 and 4 for outcome $<$, $>$, and $=$, respectively.
If the experiment results in $<$ or $>$, we know that the counterfeit coin
is coin 1 or coin 2. If it results in $=$, the counterfeit coin is coin 3 or coin 4.
Therefore, every outcome fixes two variables.

Second, consider an experiment of weighing coins 1 and 2 agains coins 3 and 4.
As exactly one coin must be counterfeit, outcome $=$ is not possible.
Outcomes $<$ and $>$ are symmetrical, both have $4$ models and fix no variables.

\autoref{tbl:run-oslas} shows the values of the defined
  strategies on these two experiments.
The experiment $12$ wins with all strategies except for Maximal number of models,
  where the values are the same.
\begin{table}[ht]
\begin{center}
\begin{tabular}{l|cc}
& 12 & 1234 \\\hline
Maximal number of models & 4 & 4 \\
Expected number of models & 3 & 4 \\
Entropy of the number of models & -1.04 & -0.69 \\
Number of satisfiable outcomes & -3 & -2 \\
Minimal number of fixed variables & -2 & 0 \\
Expected number of fixed variables & -2 & 0 \\
\end{tabular}
\caption{Values of various one-step look-ahead strategies on experiments 12 and 1234.}
\label{tbl:run-oslas}
\end{center}
\end{table}

\end{example}

% Greedy strategies are optimal in the fake-coin game $\mathcal{F}_n$.

% \TODO{Napsat důkaz.}



\section{Symmetries in Code Braking Games}

% The number of possible parametrizations of a type of experiment is typically very large,
%   which makes the analysis a game much harder.
% For example, consider the conterfeit-coin problem (\ref{prob-coins})
%   and the experiment of weighting 4 coins against 4 coins.
% There is $\frac{1}{2}\cdot {12 \choose 4}\cdot{8 \choose 4} = 17325$
%  possible combinations for parametrization, but
%  in the initial state (i.e. with no knowledge except for the initial restriction),
%  all of them are equivelent -- they will give us the same information regardless of symmetries.

% In this section, we formally define equivalence of two experiments
%  and show that we can neglect all but one experiment in each equivelence class.
% Further, we present several lemmas that will form the basis of
%  our symmetry breaking algorithms in the Chapter \ref{chapter-cobra}.

\begin{definition}[Symmetric experiment]
For an experiment $\exp = (\expt, \param)$ and a permutation $\perm\in\Perm_\Var$,
  a $\perm$-symmetric experiment $\exp^\perm = (\expt, \param')\in\Exp$
  is an experiment of the same type such that
  $\{\form^\perm \in\outcome(\exp)\} = \{\form\in\outcome(\exp^\perm)\}$.
Clearly, no such experiment may exists.
\end{definition}

\begin{definition}[Symmetry group]
We define a \emph{symmetry group} $\symg$ as
  the maximal subset of $\Perm_\Var$ such that for
  every $\perm\in\symg$ and for every experiment $\exp\in\Exp$,
  there exists a $\perm$-symmetric experiment $\exp^\perm$.
\end{definition}

\begin{definition}[Consistent strategy]
A memory-less strategy $\stg$ is \emph{consistent} if and only if
  for every $\form\in\Form_\Var$ and every $\perm\in\symg$, there
  exists $\permx\in\symg$ such that $\form^\perm \equiv \form^\permx$ and
  $\stg(\form^\permx) = \stg(\form)^\permx$.
\end{definition}

% \TODO{Example, na kterém bude vidět, že jednoduchá definice nevyhovuje, protože můžu vzít symetrie $\form$ a dostanu, že to má dávat různé věci.}

\begin{lemma}
Let $\stg$ be a memory-less strategy.
There exists a consistent memory-less strategy $\stgx$ such that
  $\stglen{\stg}{\val} >= \stglen{\stgx}{\val}$
  for all $\val\in\Val_X$ satisfying $\init$.
\end{lemma}

\begin{proof}
% Similar to the proof of Lemma \ref{lma-opt-memoryless}.
% Let us choose any total order $\form_1, \form_2, ...$ of $\Form_\Var$ such that
% if $\form_i$ implies $\form_j$, then $i <= j$.
% We build a sequence of strategies $\stg_0, \stg_1, \stg_2, ...$ in the following way:
% Let $\stg_0 = \stg$ and for $i > 0,$
% \begin{equation}
% \stg_i(\form) = \left\{
%  \begin{array}{lll}
%  \stg_{i-1}(\form) & \textrm{ if } \not\exists\perm\in\symg.\form^\perm\equiv\form_i \\
%  \procstg{\stg_{i-1}}{\val_i}(k_i + 1) &
%     \textrm{ if } \exists\perm\in\symg.\form^\perm\equiv\form_i\textrm{, where }
%     (v_i, k_i) = \argmin_K \stglen{\stg_{i-1}}{v} - k,\\
%     & \textrm{ and }
%     K = \{(v, k)\in\Val\times\Nseto \| \exists\perm.\stgknow{\stg_{i-1}}{v}{k}^\perm\equiv\form_i\}.
%     \vspace{-2.5ex}
%  \end{array}
%  \right.
%  \vspace{2.5ex}
% \end{equation}
% We prove that $\stglen{\stg_{i}}{v} <= \stglen{\stg_{i-1}}{v}$.
% If there is no $k$ and $\perm$ such that $\stgknow{\stg_i}{v}{k}^\perm\equiv\form_i$ then
%   the processes $\procstg{\stg_i}{v}$ and $\procstg{\stg_{i-1}}{v}$ are the same.
% If the is such $k$ and \TODO{...}.

% The last strategy of the sequence is consistent and satisfies the
%   condition in the lemma. \qed
\end{proof}

\begin{definition}[Experiment equivalence]
An experiment $\exp_1\in\Exp$ is equivalent to $\exp_2\in\Exp$ with respect to $\form$,
  written $\exp_1\expeq{\form}\exp_2$,
  if and only if there exists a permutation $\perm\in\symg$ such that
 $ \{ \form\wedge\formx \| \formx\in\outcome(\exp_1) \} \equiv
   \{ (\form\wedge\formx)^\perm \| \formx\in\outcome(\exp_2) \} $.
\end{definition}

\begin{theorem}
Let $\stg, \stgx$ be two consistent memory-less strategies, such that
$\stg(\form) \expeq{\form} \stgx(\form)$ for any $\form\in\Form_\Var$.
There is a bijection $f:\Val_\Var -> \Val_\Var$ such that
$\stglen{\stg}{\val} = \stglen{\stgx}{f(\val)}$.
\end{theorem}

\begin{proof}
% First, we prove by induction for any $k\in\Nseto$,
%   there is a permutation $\perm\in \symg$ such that
%   $(\stgknow{\stg}{\val}{i})^\perm = \stgknow{\stgx}{\val^\perm}{i}$
%   for all $i\in\Nseto$, $i<=k$.
% For better readability, let
%   $\know_k = \stgknow{\stg}{\val}{k}$ and
%   $\knowx_{k, \perm} = \stgknow{\stgx}{\val^\perm}{k}$

% For $k=0$, take $\perm = \idperm_\Var$.
% Clearly, $\stgknow{\stg}{\val}{0} = \init = \stgknow{\stgx}{\val^\idperm}{0}$.

% For the induction step, suppose we have $\perm\in\symg$ such that
%   $\know_i^\perm = \knowx_{i, \perm}$ for $i <= k$.
% Further, suppose $\perm$ is such that
%   $\stg(\know_k^\perm) = \stg(\know_k)^\perm$. \TODO{Víc zdůvodnit.}

% Let $e_1 = \stg(\know_k)$, $e_2 = \stgx(\knowx_{k, \perm})$
%   be the $(k+1)$-th experiments  of the strategies.
% It holds
% \begin{equation}
% e_2 = \stgx(\knowx_{k, \perm})
%     \expeq{\knowx_{k, \perm}}  \stg(\knowx_{k, \perm})
%     \stackrel{IH}{=} \stg(\know_k^\perm)
%     = \stg(\know_k)^\perm
%     = e_1^\perm \label{eq:expsym}\tag{$\sim$}
% \end{equation}

% and, therefore, there exists $\permx\in\symg$ such that
% \begin{align}
%  \{ \knowx_{k, \perm} \wedge \formx \| \formx\in\outcome(\exp_2) \} &= %\stackrel{(\refeq{eq:expsym})}{=}
%  \{ (\knowx_{k, \perm} \wedge \formx)^\permx \| \formx\in\outcome(\exp_1^\perm) \} = \label{eq:sets}\tag{*}\\
% &= \{ (\know_k^\perm \wedge \formx^\perm)^\permx \| \formx\in\outcome(\exp_1) \} =
%  \{ (\know_k \wedge \formx)^{\permx\perm} \| \formx\in\outcome(\exp_1) \}
% \end{align}
% As $\permx\in\symg$ and $\symg$ is a permutation group, $\permx\perm\in\symg$.

% Since the game is well-formed,
%   $v$ satisfies exactly one formula in
%   $\{ \know_k \wedge \formx \| \formx\in\outcome(\exp_1) \}$.
% Therefore $v^{\permx\perm}$ satisfies exactly one formula
%   in
%   $\{ (\know_k \wedge \formx)^{\permx\perm} \| \formx\in\outcome(\exp_1) \}  =
%    \{ \knowx_{k, \perm} \wedge \formx \| \formx\in\outcome(\exp_2) \}$,
%   which means that $v^{\permx\perm}$ satisfies $\knowx_{k, \perm}$.
% \TODO{Tohle nefugujeee!}
% From Lemma \ref{lma-accruedknowledge},
%   $\knowx_{k, \perm} = \knowx_{k, \permx\perm}$.
% Both $\know_{k+1}^{\permx\perm}$ and $\knowx_{k+1, \permx\perm}$ is thus the only
%   formula from (\refeq{eq:sets}) satisfied by $v^{\permx\perm}$ and, therefore,
%   $\know_{k+1}^{\permx\perm} = \knowx_{k+1, \permx\perm}$.



% Now for a fixed $\val$, take $k = \stglen{\stg}{\val}$, take
%   $\perm\in\symg$ such that
%   $(\stgknow{\stg}{\val}{k})^\perm = \stgknow{\stgx}{\val^\perm}{k}$
%   and define $f(\val) = \val^\perm$.
% Since
%  $(\stgknow{\stg}{\val}{i})^\perm = \stgknow{\stgx}{f(\val)}{i}$
%  for $i <= k$
%  and variable permutation preserves the number of models of a formula, i.e.
%   $\numval{\form} = \numval{\form^\perm}$ for any
%   $\form\in\Form_\Var$, $\perm\in\Perm_\Var$,
%  we have
%   $\stglen{\stg}{\val} = \stglen{\stgx}{f(\val)}$.
% % \TODO{It remains to show that $f$ is a bijection.}
%   % Suppose $f$ is not injective, and $f(\val_1) = f(\val_2)$.
%   % By definition, the only model of
%   \qed
\end{proof}

\begin{corollary}
Let $\stg_1, \stg_2$ be two consistent memory-less strategies, such that
  $\stg_1(\form) \expeq{\form} \stg_2(\form)$ for any $\form\in\Form_\Var$.
Then $\lenmax{\stg_1} = \lenmax{\stg_2}$
  and $\lenexp{\stg_1} = \lenexp{\stg_2}$.
\end{corollary}

% For any accrued knowledge $\form$, this lemma gives us the right
% to consider only one of the experiments
% $\exp_1, \exp_2$ if $\exp_1 \sim_\form \exp_2$.

% Now, we would love an algorithm that would, for a given formula $\form$,
% generate a set of experiment, such that there is exaclty one expriment
% from every equivalence class in $E/\sim_\form$.

% \TODO{...}

% For the following sections, let us fix an experiment type $t$.
% \subsection{Phase 1}

% \subsection{Phase 2}
\section{Symmetry Breaking}

\subsection{Phase 1 - Interchangeble symbols}

\subsection{Phase 2 - Canonical Form of parametrization}

\subsection{Phase 3 - Canonical Form of formula graph}

\subsection{Comparison}

